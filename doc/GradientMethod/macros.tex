% !TEX root = FemM1.tex
%----------------------------------------


%----------------------------------------

\definecolor{myyellow}{rgb}{0.9, 0.9, 0.01}
\definecolor{myblue}{rgb}{0.004, 0.1, 0.57}
\definecolor{myred}{rgb}{0.58, 0.066, 0.}
\definecolor{mygreen}{rgb}{0.24, 0.55, 0.15}
\definecolor{mygray}{rgb}{0.4, 0.6, 0.6}
\definecolor{myviolet}{rgb}{0.3, 0.1, 0.4}
\definecolor{myorange}{rgb}{0.4654205607476635, 0.33271028037383177, 0.20186915887850468}
\newcommand{\blue}[1]{\textcolor{myblue}{#1}}
\newcommand{\red}[1]{\textcolor{myred}{#1}}
\newcommand{\green}[1]{\textcolor{mygreen}{#1}}
\newcommand{\gray}[1]{\textcolor{mygray}{#1}}
\newcommand{\violet}[1]{\textcolor{myviolet}{#1}}
\definecolor{greenlight}{rgb}{0.95,1,0.95}
\definecolor{bluelight}{rgb}{0.9,0.9,1}
\definecolor{graylighy}{rgb}{0.975,0.975,0.975}
\definecolor{yellowlight}{rgb}{0.999, 0.999, 0.925}
\definecolor{roselight}{rgb}{0.99, 0.9, 0.95}
%
\newenvironment{blueenv}{\color{myblue}}{}
\newenvironment{greenenv}{\color{mygreen}}{}
\newenvironment{grayenv}{\color{mygray}}{}
\newenvironment{yellow}[1][]{\begin{tcolorbox}[breakable,title={#1} , boxrule=2pt, colback=yellowlight]}{\end{tcolorbox}}
\newenvironment{rose}[1][]{\begin{tcolorbox}[breakable,title={#1} , boxrule=2pt, colback=roselight]}{\end{tcolorbox}}



%---------------------------------------------------------
%
\newtheorem{theorem}{Theorem}
\newtheorem{lemma}{Lemma}
\newtheorem{corollary}{Corollary}
\newtheorem{proposition}{Proposition}
\newtheorem{remark}{Remark}
\newtheorem{example}{Example}
%

%----------------------------------------
\newcommand{\dpfrac}[2]{\frac{\partial #1}{\partial #2}} 
\newcommand{\ddpfracdiag}[2]{\frac{\partial^2 #1}{\partial #2^2}} 
\newcommand{\ddpfrac}[3]{\frac{\partial^2 #1}{\partial #2\partial #3}} 
\newcommand{\dddpfrac}[4]{\frac{\partial^3 #1}{\partial #2\partial #3\partial #4}} 
\newcommand{\dn}[1]{\dpfrac{#1}{n}} 
\newcommand{\dpt}[1]{\dpfrac{#1}{t}} 
\newcommand{\dptt}[1]{\frac{\partial^2 #1}{\partial t^2 }}  
\newcommand{\dbeta}[1]{\dpfrac{#1}{\beta}} 

%----------------------------------------
% OPT
%\newcommand{\argmin}{\operatorname{argmin}}
\DeclareMathOperator*{\argmin}{argmin}
\DeclareMathOperator*{\argmax}{argmax}

%---------------------------------------------------------
%
% AFEM
%
\newcommand{\REF}{{\mbox{\textbf{REF}}}}
\newcommand{\MARK}{{\mbox{\textbf{MARK}}}}
\newcommand{\allmeshes}{{\mathbb H}}
\newcommand{\meshcriterion}{\mathcal C}
%
\newcommand{\Csolve}{C_{\rm s}}
\newcommand{\Cmesh}{C_{\rm mesh}}
\newcommand{\Cclose}{C_{\rm cls}}
\newcommand{\Cref}{C_{\rm ref}}
\newcommand{\Cglobrel}{C_{\rm gr}}
\newcommand{\Cloceff}{C_{\rm le}}
\newcommand{\Copt}{C_{\rm opt}}
\newcommand{\Cneighb}{C_{\rm nei}}
\newcommand{\Cmeshopt}{C_{\rm mopt}}
\newcommand{\Cgeom}{C_{\rm g}}
\newcommand{\Cmeshcrit}{C_{\rm mc}}
\newcommand{\Cmon}{C_{\rm mon}}
\newcommand{\Cemon}{C_{\rm em}}
\newcommand{\Ceststab}{C_{\rm es}}
\newcommand{\Cinv}{C_{\rm inv}}
\newcommand{\qmesh}{q_{\rm m}}
\newcommand{\qsolve}{q_{\rm s}}

\newcommand{\chistop}[1]{{\chi_{#1}^{\rm fin}}}
\newcommand{\DeltaStop}[1]{{\widetilde{\Delta}_{#1}}}
\newcommand{\solvecrit}[1]{{\mathcal S_{#1}}}
\newcommand{\qred}{q_{\rm red}}
\newcommand{\Crel}{C_{\rm rel}}
\newcommand{\Cstab}{C_{\rm stab}}
\newcommand{\Ceff}{C_{\rm eff}}

\newcommand{\Cupp}{C_{\rm up}}
\newcommand{\Clow}{C_{\rm low}}
\newcommand{\true}{{\mathtt{true}}}
\newcommand{\false}{{\mathtt{false}}}

%---------------------------------------------------------
%
\newcommand{\Expt}[1]{{\rm E}(#1)}
\newcommand{\Var}[1]{{\rm Var}(#1)}
%
\renewcommand{\Re}{\rm Re} 
\renewcommand{\Im}{\rm Im}
\newcommand{\N}{\mathbb N}
\newcommand{\R}{\mathbb R}
\newcommand{\Z}{\mathbb Z}
\newcommand{\C}{\mathbb C}
%
% 
\newcommand{\Set}[1]{ \left\{#1\right\}}
\newcommand{\SetDef}[2]{\left\{#1\;\middle|\;#2\right\}} 
\newcommand{\Rest}[2]{{#1}_{|_{#2}}}
\newcommand{\transpose}[1]{#1^{\mathsf{T}}} 
\newcommand{\transposeInv}[1]{#1^{\mathsf{-T}}} 
%
\newcommand{\supp}[1]{\operatorname{supp}\left(#1\right)}
\renewcommand{\div}{\operatorname{div}}
\newcommand{\rot}{\operatorname{rot}}
\newcommand{\diag}{\operatorname{diag}}
\newcommand{\norm}[1]{\left\|#1\right\|}
\makeatletter
\newcommand{\opnorm}{\@ifstar\@opnorms\@opnorm}
\newcommand{\@opnorms}[1]{%
  \left|\mkern-1.5mu\left|\mkern-1.5mu\left|
   #1
  \right|\mkern-1.5mu\right|\mkern-1.5mu\right|
}
\newcommand{\@opnorm}[2][]{%
  \mathopen{#1|\mkern-1.5mu#1|\mkern-1.5mu#1|}
  #2
  \mathclose{#1|\mkern-1.5mu#1|\mkern-1.5mu#1|}
}
\makeatother
\newcommand{\abs}[1]{\left\vert{#1}\right\vert}
\newcommand{\tnorm}[1]{\opnorm{#1}}
\newcommand{\eps}{\varepsilon}
\newcommand{\scp}[2]{\langle#1,#2\rangle}
\newcommand{\Pe}{\mathrm {Pe}}
\newcommand{\Sides}{\mathcal S}
\newcommand{\Cells}{\mathcal K}
\newcommand{\Nodes}{\mathcal N}
\newcommand{\NodesInt}{{\mathcal N}^{int}}
\newcommand{\level}{\operatorname{lev}}
\newcommand{\SidesInt}{\mathcal S^{\rm int}}
\newcommand{\SidesBdry}{\mathcal S^{\partial}}
\newcommand{\Kin}{K^{\rm{\footnotesize{in}}}}
\newcommand{\Kex}{K^{\rm{\footnotesize{ex}}}}
\newcommand{\inS}[1]{{#1}^{\rm{\footnotesize{in}}}_{S}}
\newcommand{\exS}[1]{{#1}^{\rm{\footnotesize{ex}}}_{S}}
\newcommand{\inSs}[1]{{#1}^{\rm{\footnotesize{in}}}}
\newcommand{\exSs}[1]{{#1}^{\rm{\footnotesize{ex}}}}
\newcommand{\meanS}[1]{\left\{\{#1\right\}_{S}}
\newcommand{\jumpS}[1]{\left[#1\right]_{S}}
\newcommand{\jump}[1]{\left[#1\right]}
\newcommand{\mean}[1]{\left\{#1\right\}}
\newcommand{\intS}{\int_{\Sides_{h}}}
\newcommand{\intSInt}{\int_{\SidesInt_{h}}}
\newcommand{\intSBound}{\int_{\SidesBound_{h}}}
%\newcommand{\intK}{\int_{\Cells_{h}}}
\newcommand{\REFLOC}{{\rm REF}}
\newcommand{\AFEM}[1]{\textbf{AFEM}($#1$)}

%
\newcommand{\tesfctu}{\delta\! u}
\newcommand{\udir}{u^{\rm D}}
\newcommand{\vdir}{v^{\rm D}}
\newcommand{\wdir}{w^{\rm D}}
\newcommand{\uh}{u_{h}}
\newcommand{\phih}{\phi_{h}}
\newcommand{\uin}{u^{\rm in}}
\newcommand{\uex}{u^{\rm ex}}
\newcommand{\uhin}{u^{\rm in}_{h}}
\newcommand{\uhex}{u^{\rm ex}_{h}}
\newcommand{\ubar}{\overline{u}}
\newcommand{\ustar}{u^{*}}
\newcommand{\sgn}[1]{\operatorname{sgn}(#1)}
%
\newcommand{\In}[1]{#1^{\rm in}}
\newcommand{\Ex}[1]{#1^{\rm ex}}
\newcommand{\InDe}[2]{#1^{{\rm in}_{#2}}}
\newcommand{\ExDe}[2]{#1^{{\rm ex}_{#2}}}
