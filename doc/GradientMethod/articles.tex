%----------------------------------------
\documentclass[12pt,english]{article}
%----------------------------------------
%
%---------------------------------------------------------
\usepackage[a4paper,margin=1.8cm]{geometry}
% !TEX root = FemM1.tex
%----------------------------------------


%----------------------------------------
\usepackage{amsmath,amssymb,amsthm,latexsym}
%\usepackage{stmaryrd}
%\usepackage{pifont}
\usepackage{mathtools}
%\usepackage{refcheck}
\usepackage{hyperref}
%%----------------------------------------
%\usepackage[utf8]{inputenc}
\usepackage{palatino,eulervm}
%---------------------------------------------------------
\usepackage{url}
\usepackage[english,algoruled,lined]{algorithm2e}
\usepackage{listings}
\usepackage{xcolor}
\usepackage[many]{tcolorbox}
\tcbuselibrary{breakable}
\usepackage{xcolor}
\usepackage{cancel}
\usepackage{mathtools} 
\usepackage{enumitem}

%---------------------------------------------------------

%---------------------------------------------------------
% !TEX root = FemM1.tex
%----------------------------------------


%----------------------------------------

\definecolor{myyellow}{rgb}{0.9, 0.9, 0.01}
\definecolor{myblue}{rgb}{0.004, 0.1, 0.57}
\definecolor{myred}{rgb}{0.58, 0.066, 0.}
\definecolor{mygreen}{rgb}{0.24, 0.55, 0.15}
\definecolor{mygray}{rgb}{0.4, 0.6, 0.6}
\definecolor{myviolet}{rgb}{0.3, 0.1, 0.4}
\definecolor{myorange}{rgb}{0.4654205607476635, 0.33271028037383177, 0.20186915887850468}
\newcommand{\blue}[1]{\textcolor{myblue}{#1}}
\newcommand{\red}[1]{\textcolor{myred}{#1}}
\newcommand{\green}[1]{\textcolor{mygreen}{#1}}
\newcommand{\gray}[1]{\textcolor{mygray}{#1}}
\newcommand{\violet}[1]{\textcolor{myviolet}{#1}}
\definecolor{greenlight}{rgb}{0.95,1,0.95}
\definecolor{bluelight}{rgb}{0.9,0.9,1}
\definecolor{graylighy}{rgb}{0.975,0.975,0.975}
\definecolor{yellowlight}{rgb}{0.999, 0.999, 0.925}
\definecolor{roselight}{rgb}{0.99, 0.9, 0.95}
%
\newenvironment{blueenv}{\color{myblue}}{}
\newenvironment{greenenv}{\color{mygreen}}{}
\newenvironment{grayenv}{\color{mygray}}{}
\newenvironment{yellow}[1][]{\begin{tcolorbox}[breakable,title={#1} , boxrule=2pt, colback=yellowlight]}{\end{tcolorbox}}
\newenvironment{rose}[1][]{\begin{tcolorbox}[breakable,title={#1} , boxrule=2pt, colback=roselight]}{\end{tcolorbox}}



%---------------------------------------------------------
%
\newtheorem{theorem}{Theorem}
\newtheorem{lemma}{Lemma}
\newtheorem{corollary}{Corollary}
\newtheorem{proposition}{Proposition}
\newtheorem{remark}{Remark}
\newtheorem{example}{Example}
%

%----------------------------------------
\newcommand{\dpfrac}[2]{\frac{\partial #1}{\partial #2}} 
\newcommand{\ddpfracdiag}[2]{\frac{\partial^2 #1}{\partial #2^2}} 
\newcommand{\ddpfrac}[3]{\frac{\partial^2 #1}{\partial #2\partial #3}} 
\newcommand{\dddpfrac}[4]{\frac{\partial^3 #1}{\partial #2\partial #3\partial #4}} 
\newcommand{\dn}[1]{\dpfrac{#1}{n}} 
\newcommand{\dpt}[1]{\dpfrac{#1}{t}} 
\newcommand{\dptt}[1]{\frac{\partial^2 #1}{\partial t^2 }}  
\newcommand{\dbeta}[1]{\dpfrac{#1}{\beta}} 

%----------------------------------------
% OPT
%\newcommand{\argmin}{\operatorname{argmin}}
\DeclareMathOperator*{\argmin}{argmin}
\DeclareMathOperator*{\argmax}{argmax}

%---------------------------------------------------------
%
% AFEM
%
\newcommand{\REF}{{\mbox{\textbf{REF}}}}
\newcommand{\MARK}{{\mbox{\textbf{MARK}}}}
\newcommand{\allmeshes}{{\mathbb H}}
\newcommand{\meshcriterion}{\mathcal C}
%
\newcommand{\Csolve}{C_{\rm s}}
\newcommand{\Cmesh}{C_{\rm mesh}}
\newcommand{\Cclose}{C_{\rm cls}}
\newcommand{\Cref}{C_{\rm ref}}
\newcommand{\Cglobrel}{C_{\rm gr}}
\newcommand{\Cloceff}{C_{\rm le}}
\newcommand{\Copt}{C_{\rm opt}}
\newcommand{\Cneighb}{C_{\rm nei}}
\newcommand{\Cmeshopt}{C_{\rm mopt}}
\newcommand{\Cgeom}{C_{\rm g}}
\newcommand{\Cmeshcrit}{C_{\rm mc}}
\newcommand{\Cmon}{C_{\rm mon}}
\newcommand{\Cemon}{C_{\rm em}}
\newcommand{\Ceststab}{C_{\rm es}}
\newcommand{\Cinv}{C_{\rm inv}}
\newcommand{\qmesh}{q_{\rm m}}
\newcommand{\qsolve}{q_{\rm s}}

\newcommand{\chistop}[1]{{\chi_{#1}^{\rm fin}}}
\newcommand{\DeltaStop}[1]{{\widetilde{\Delta}_{#1}}}
\newcommand{\solvecrit}[1]{{\mathcal S_{#1}}}
\newcommand{\qred}{q_{\rm red}}
\newcommand{\Crel}{C_{\rm rel}}
\newcommand{\Cstab}{C_{\rm stab}}
\newcommand{\Ceff}{C_{\rm eff}}

\newcommand{\Cupp}{C_{\rm up}}
\newcommand{\Clow}{C_{\rm low}}
\newcommand{\true}{{\mathtt{true}}}
\newcommand{\false}{{\mathtt{false}}}

%---------------------------------------------------------
%
\newcommand{\Expt}[1]{{\rm E}(#1)}
\newcommand{\Var}[1]{{\rm Var}(#1)}
%
\renewcommand{\Re}{\rm Re} 
\renewcommand{\Im}{\rm Im}
\newcommand{\N}{\mathbb N}
\newcommand{\R}{\mathbb R}
\newcommand{\Z}{\mathbb Z}
\newcommand{\C}{\mathbb C}
%
% 
\newcommand{\Set}[1]{ \left\{#1\right\}}
\newcommand{\SetDef}[2]{\left\{#1\;\middle|\;#2\right\}} 
\newcommand{\Rest}[2]{{#1}_{|_{#2}}}
\newcommand{\transpose}[1]{#1^{\mathsf{T}}} 
\newcommand{\transposeInv}[1]{#1^{\mathsf{-T}}} 
%
\newcommand{\supp}[1]{\operatorname{supp}\left(#1\right)}
\renewcommand{\div}{\operatorname{div}}
\newcommand{\rot}{\operatorname{rot}}
\newcommand{\diag}{\operatorname{diag}}
\newcommand{\norm}[1]{\left\|#1\right\|}
\makeatletter
\newcommand{\opnorm}{\@ifstar\@opnorms\@opnorm}
\newcommand{\@opnorms}[1]{%
  \left|\mkern-1.5mu\left|\mkern-1.5mu\left|
   #1
  \right|\mkern-1.5mu\right|\mkern-1.5mu\right|
}
\newcommand{\@opnorm}[2][]{%
  \mathopen{#1|\mkern-1.5mu#1|\mkern-1.5mu#1|}
  #2
  \mathclose{#1|\mkern-1.5mu#1|\mkern-1.5mu#1|}
}
\makeatother
\newcommand{\abs}[1]{\left\vert{#1}\right\vert}
\newcommand{\tnorm}[1]{\opnorm{#1}}
\newcommand{\eps}{\varepsilon}
\newcommand{\scp}[2]{\langle#1,#2\rangle}
\newcommand{\Pe}{\mathrm {Pe}}
\newcommand{\Sides}{\mathcal S}
\newcommand{\Cells}{\mathcal K}
\newcommand{\Nodes}{\mathcal N}
\newcommand{\NodesInt}{{\mathcal N}^{int}}
\newcommand{\level}{\operatorname{lev}}
\newcommand{\SidesInt}{\mathcal S^{\rm int}}
\newcommand{\SidesBdry}{\mathcal S^{\partial}}
\newcommand{\Kin}{K^{\rm{\footnotesize{in}}}}
\newcommand{\Kex}{K^{\rm{\footnotesize{ex}}}}
\newcommand{\inS}[1]{{#1}^{\rm{\footnotesize{in}}}_{S}}
\newcommand{\exS}[1]{{#1}^{\rm{\footnotesize{ex}}}_{S}}
\newcommand{\inSs}[1]{{#1}^{\rm{\footnotesize{in}}}}
\newcommand{\exSs}[1]{{#1}^{\rm{\footnotesize{ex}}}}
\newcommand{\meanS}[1]{\left\{\{#1\right\}_{S}}
\newcommand{\jumpS}[1]{\left[#1\right]_{S}}
\newcommand{\jump}[1]{\left[#1\right]}
\newcommand{\mean}[1]{\left\{#1\right\}}
\newcommand{\intS}{\int_{\Sides_{h}}}
\newcommand{\intSInt}{\int_{\SidesInt_{h}}}
\newcommand{\intSBound}{\int_{\SidesBound_{h}}}
%\newcommand{\intK}{\int_{\Cells_{h}}}
\newcommand{\REFLOC}{{\rm REF}}
\newcommand{\AFEM}[1]{\textbf{AFEM}($#1$)}

%
\newcommand{\tesfctu}{\delta\! u}
\newcommand{\udir}{u^{\rm D}}
\newcommand{\vdir}{v^{\rm D}}
\newcommand{\wdir}{w^{\rm D}}
\newcommand{\uh}{u_{h}}
\newcommand{\phih}{\phi_{h}}
\newcommand{\uin}{u^{\rm in}}
\newcommand{\uex}{u^{\rm ex}}
\newcommand{\uhin}{u^{\rm in}_{h}}
\newcommand{\uhex}{u^{\rm ex}_{h}}
\newcommand{\ubar}{\overline{u}}
\newcommand{\ustar}{u^{*}}
\newcommand{\sgn}[1]{\operatorname{sgn}(#1)}
%
\newcommand{\In}[1]{#1^{\rm in}}
\newcommand{\Ex}[1]{#1^{\rm ex}}
\newcommand{\InDe}[2]{#1^{{\rm in}_{#2}}}
\newcommand{\ExDe}[2]{#1^{{\rm ex}_{#2}}}

%---------------------------------------------------------

%-------------------------------------------
\begin{document}
%-------------------------------------------

\title{Articles on Gradient methods}
\author{}
\maketitle
\tableofcontents

%
%==========================================
\section{OC15}\label{sec:}
%==========================================
%
From \cite{ODonoghueCandes15}.

They use AGM in the form: $\theta_0=1$ and  $\theta_k$ solves

\begin{align*}
\theta_{k+1}^2 = (1-\theta_{k+1})\theta_k^2+ q\theta_{k+1}\\
\beta_k = \theta_k(1-\theta_k)/(\theta_k^2+\theta_{k+1}) \\
y_{k+1} = x_{k+1} + \beta_k(x_{k+1}-x_{k})
\end{align*}
%
For $q=1$ we have the GM.


%
\begin{align*}
\theta_{k+1}^2 + (\theta_k^2- q)\theta_{k+1}= \theta_k^2
\quad\Leftrightarrow\quad 
\left( \theta_{k+1} + \frac{\theta_k^2- q}{2}\right)^2 = \theta_k^2 + \frac{(\theta_k^2- q)^2}{4}\\
\quad\Leftrightarrow\quad 
\theta_{k+1} = \sqrt{\theta_k^2 + \frac{(\theta_k^2- q)^2}{4}}-\frac{\theta_k^2- q}{2}
\end{align*}
%

%
%-------------------------------------------------------------------------
\subsection{Observation: dependence on $q$}\label{subsec:}
%-------------------------------------------------------------------------
%
... is impressive.
%
%-------------------------------------------------------------------------
\subsection{Restart}\label{subsec:}
%-------------------------------------------------------------------------
%
Restart rules:
%
\begin{align*}
f(x_{k+1}) > f(x_k)\\
\scp{\nabla f(y_k)}{x_{k+1}-x_{k}} >0
\end{align*}
%
%
%-------------------------------------------------------------------------
\subsection{Linear convergence analysis}\label{subsec:}
%-------------------------------------------------------------------------
%
For $f(x) = \frac12 \transpose{x} A x$. And even $n=1$, $A=\lambda$.

Suppose
%
\begin{align*}
%
\left\{
\begin{aligned}
x_{k+1} = y_k - \frac{1}{L}\nabla f(y_k)\\
y_{k+1} = x_{k+1} + \beta(x_{k+1}-x_{k})
\end{aligned}
\right.
\quad\Rightarrow\quad 
x_{k+1} =  (1-\frac{\lambda}{L}) \left( (1+\beta)x_{k} - \beta(x_{k-1})\right)
\end{align*}
%
The iteration is governed by the characteristic polynomial
%
\begin{align*}
r^2 - (1-\frac{\lambda}{L}) \left( (1+\beta)r - \beta\right)
\end{align*}
%
Minimizing the module of the roots $\abs{r^*}$ gives
%
\begin{align*}
\beta^* = \frac{1-\sqrt{\lambda/L}}{1+\sqrt{\lambda/L}}\quad\Rightarrow\quad 
\abs{r^*} = 1 -\sqrt{\lambda/L}.
\end{align*}
%
For $\beta<\beta^*$ we are in the low momentum regime, and we say the system is over- damped. The convergence rate is dominated by the larger root, i.e., the system exhibits slow monotone convergence.
If  $\beta>\beta^*$ then the roots of the polynomial (7) are complex and we are in the high momentum regime. The system is under-damped and exhibits periodicity.

%
%==========================================
\section{AZO14}\label{sec:}
%==========================================
%
From \cite{Allen-ZhuOrecchia14}. They use a strongly convex distance generating function (DGF) $\psi$ and corresponding Bregman divergence $\Delta_{\psi}(x,y) := \psi(x)-\psi(y)-\nabla\psi(y)(x-y)$.

We only consider the Euclidian norm. Then in our notation the considered alogorithm reads
%
\begin{align*}
y_0=&z_0=x_0\\
y_{k} =& (1-\tau_k) x_k + \tau_k z_k\\
x_{k+1} =& y_k - t_k \nabla f(y_k)\\
z_{k+1} =& z_k - \alpha_k  \nabla f(y_k)\\
\tau_k =& t_k/\alpha_k\\
t_k =& 1/L,\quad \alpha_k=(k+2)/(2L) 
\end{align*}
%
We then have
%
\begin{align*}
z_{k+1} = z_k + \frac{\alpha_k}{t_k}  (x_{k+1} - y_k)
= z_k + \frac{\alpha_k}{t_k}  (x_{k+1} - ((1-\tau_k) x_k + \tau_k z_k))
= x_{k+1} + \frac{1-\tau_k}{\tau_k}(x_{k+1}-x_k)
\end{align*}
%
and
%
\begin{align*}
y_{k} =& (1-\tau_k) x_k + \tau_k (x_{k} + \frac{1-\tau_{k-1}}{\tau_{k-1}}(x_{k}-x_{k-1}))
=  x_k  + \frac{\tau_k(1-\tau_{k-1})}{\tau_{k-1}}(x_{k}-x_{k-1})
\end{align*}
%
%
%==========================================
\section{DFR18}\label{sec:}
%==========================================
%
From \cite{DrusvyatskiyFazelRoy18a}, inspired by \cite{BubeckLeeSingh15}.

%---------------------------------------
\begin{lemma}(Quadratic Averaging)\label{lemma:quadav}
Let $Q_i(x) = Q_i^* + \frac{\alpha}{2}\norm{x-c_i}^2$ and $Q(\lambda, x) = (1-\lambda)Q_1(x)+\lambda Q_2(x)$. Then
%
\begin{equation}\label{eq:qa_sol}
%
\left\{
\begin{aligned}
\max_{0\le\lambda\le1} Q^*(\lambda) =& (1-\lambda^*)Q_1^*+\lambda^* Q_2^* + \frac{\lambda^*(1-\lambda^*)\alpha}{2}\norm{c_1-c_2}^2,\\ 
\argmax_{0\le\lambda\le1} Q^*(\lambda) =& (1-\lambda^*)c_1+\lambda^* c_2,\\
\lambda^* =& P_{[0;1]}\left(\frac12 + \frac{(Q_2^*-Q_1^*)}{\alpha\norm{c_1-c_2}^2}\right).
\end{aligned}
\right.
%
\end{equation}
%
If 
%
\begin{equation}\label{eq:qa_cond}
\frac{\abs{Q_2^*-Q_1^*}}{\alpha\norm{c_1-c_2}^2}\le \frac12
\end{equation}
%
we have
%
\begin{equation}\label{eq:qa_sol}
Q^*(\lambda^*) =\frac{Q_1^*+Q_2^*}{2}+ \frac{\alpha}{8}\norm{c_1-c_2}^2+\frac{(Q_2^*-Q_1^*)^2}{2\alpha\norm{c_1-c_2}^2}
\end{equation}
%
and the function $Q^*(\lambda^*)$ is nondecreasing in $Q_i^*$.
\end{lemma}
%
%---------------------------------------
\begin{proof}
We have with $a^2-2ab = (a-b)^2 -b^2$
%
\begin{align*}
(1-\lambda)\norm{x-c_1}^2 + \lambda\norm{x-c_2}^2 
=&\norm{x}^2 - 2\scp{x}{(1-\lambda)c_1+\lambda c_2}+(1-\lambda)\norm{c_1}^2 + \lambda\norm{c_2}^2 \\
=& \norm{x - (1-\lambda)c_1+\lambda c_2}^2 - \norm{(1-\lambda)c_1+\lambda c_2}^2 +(1-\lambda)\norm{c_1}^2 + \lambda\norm{c_2}^2\\
=& \norm{x - (1-\lambda)c_1+\lambda c_2}^2 +\lambda(1-\lambda)\norm{c_1-c_2}^2,
\end{align*}
%
so
%
\begin{align*}
Q(\lambda, x) =& (1-\lambda)Q_1^*+\lambda Q_2^* + \frac{(1-\lambda)\alpha}{2}\norm{x-c_1}^2 + \frac{\lambda\alpha}{2}\norm{x-c_2}^2 \\
=& (1-\lambda)Q_1^*+\lambda Q_2^* + \frac{\lambda(1-\lambda)\alpha}{2}\norm{c_1-c_2}^2
+\frac{\alpha}{2}\norm{x - (1-\lambda)c_1+\lambda c_2}^2,
\end{align*}
%
which gives
%
\begin{align*}
Q^*(\lambda) = (1-\lambda)Q_1^*+\lambda Q_2^* + \frac{\lambda(1-\lambda)\alpha}{2}\norm{c_1-c_2}^2,\quad \argmin_{x} Q(\lambda,x)=(1-\lambda)c_1+\lambda c_2
\end{align*}
%
since 
%
\begin{align*}
\frac{d Q^*(\lambda)}{d\lambda} = Q_2^*-Q_1^* + \frac{\alpha}{2}\norm{c_1-c_2}^2 -\lambda\alpha\norm{c_1-c_2}^2
\end{align*}
%
we find
%
\begin{align*}
\lambda^* = P_{[0;1]}\left(\frac12 + \frac{(Q_2^*-Q_1^*)}{\alpha\norm{c_1-c_2}^2}\right).
\end{align*}
%
If (\ref{eq:qa_cond})
%
\begin{align*}
\lambda^*=&\frac12 + \frac{Q_2^*-Q_1^*}{\alpha\norm{c_1-c_2}^2}\\
Q^*(\lambda^*) =& \frac{Q_1^*+Q_2^*}{2}+\frac{(Q_2^*-Q_1^*)}{\alpha\norm{c_1-c_2}^2}(Q_2^*-Q_1^*) + \frac{\alpha\left(\frac14-\frac{\abs{Q_2^*-Q_1^*}^2}{\alpha^2\norm{c_1-c_2}^4}\right)}{2}\norm{c_1-c_2}^2\\
=& \frac{Q_1^*+Q_2^*}{2}+ \frac{\alpha}{8}\norm{c_1-c_2}^2+\frac{(Q_2^*-Q_1^*)^2}{2\alpha\norm{c_1-c_2}^2}\\
=& Q_2^* + \frac{Q_1^*-Q_2^*}{2} + \frac{\alpha}{8}\norm{c_1-c_2}^2+\frac{(Q_2^*-Q_1^*)^2}{2\alpha\norm{c_1-c_2}^2}\\
=& Q_2^* +\frac{\left( (Q_2^*-Q_1^*)+\frac{\alpha}{2}\norm{c_1-c_2}^2\right)^2}{2\alpha\norm{c_1-c_2}^2}
\end{align*}
%
Finally we have
%
\begin{align*}
\frac{\partial Q^*(\lambda^*)}{\partial Q_1^*} = \frac12 -\frac{Q_2^*-Q_1^*}{\alpha\norm{c_1-c_2}^2}\ge0.
\end{align*}
%
\end{proof}
%
Let
%
\begin{align*}
x^+ := x - \frac{1}{L}\nabla f(x),\quad x^{++} := x - \frac{1}{\mu}\nabla f(x).
\end{align*}
%
%
%---------------------------------------
\begin{yellow}
\begin{algorithm}[H]
\caption{Quadratic averaging} 
\label{algorithm:QA} 
%
Inputs: $x_0\in X$. Set $k=0$, $v_0:= f(x_0)-\frac{\norm{\nabla f(x_0)}^2}{2\mu}$, $c_0:=x_0^{++}$, $Q_0(x) = v_0 + \frac{\mu}{2}\norm{x-c_0}^2$
%
\begin{itemize}
\item[(1)] $x_{k+1} := \min_{0\le t\le 1} (c_k + t(x_k^{+}-c_k))$.
\item[(2)] 
$\widetilde{v}:= f(x_{k+1}) - \frac{\norm{\nabla f(x_{k+1})}^2}{2\mu}$,\quad 
$\lambda_{k}:=P_{[0;1]}\left(\frac12 + \frac{v_k-\widetilde{v}}{\mu\norm{c_k-x_{k+1}^{++}}^2}\right)$,\quad $c_{k+1} := (1-\lambda_k)x_{k+1}^{++} + \lambda_k c_k$,\quad $v_{k+1} = (1-\lambda_k)\widetilde{v}+\lambda_k v_k + \frac{\lambda_k(1-\lambda_k)\mu}{2}\norm{x_{k+1}^{++}-c_k}^2$
\item[(3)] Increment $k$ and go to (1).
\end{itemize}
%
\end{algorithm}
\end{yellow}
%---------------------------------------

%---------------------------------------
\begin{theorem}(2.3)\label{thm:}
We have
%
\begin{equation}\label{eq:}
v_k \le f^* \le f(x_k^+),\quad f(x_k^+)-v_k \le \rho^k \left(f(x_0^+)-v_0\right), \quad \rho:=1-1/\sqrt{\kappa}.
\end{equation}
%
\end{theorem}
%
%---------------------------------------
\begin{proof}
Let $r_k := \rho^k \left(f(x_0^+)-v_0\right)$. By induction we show $ f(x_k^+)\le v_k+r_k$. For $k=0$ this evident. Let the induction hypothesis be true. We want to show
%
\begin{align*}
f(x_{k+1}^+) \le v_{k+1} + r_{k+1}.
\end{align*}
%

%
%---------------GRAYGREEN------------
\begin{grayenv}
%
We have
%
\begin{align*}
f(x_{k+1}^+) \le& f(x_{k+1}) - \frac{1}{2L}\norm{\nabla f(x_{k+1})}^2&& \mbox{(Lipschitz)}\\
\le& f(x_{k}^+) - \frac{1}{2L}\norm{\nabla f(x_{k+1})}^2&& \mbox{(Line-search)}\\
\le& v_k + r_k - \frac{1}{2L}\norm{\nabla f(x_{k+1})}^2&& \mbox{(Induction)}\\
\end{align*}
%
Now suppose that
%
\begin{equation}\label{eq:qa_proof_hyp}
\norm{\nabla f(x_{k+1})}^2 \ge 2 \sqrt{L\mu}r_k.
\end{equation}
%
Then
%
\begin{align*}
f(x_{k+1}^+) \le& v_k + \left(1-\frac{\sqrt{\mu}}{\sqrt{L}}\right)r_k&& \mbox{(\ref{eq:qa_proof_hyp})}\\
\le& v_{k+1} + r_{k+1}&& \mbox{($v_k$ increasing)}
\end{align*}
%

%
\end{grayenv}
%---------------GRAYGREEN------------
%
Let
%
\begin{equation}\label{eq:qa_proof_case1}
\frac{\norm{\nabla f(x_{k+1})}^2}{\mu} \le  2\frac{\sqrt{\kappa}}{\sqrt{\kappa}+1}r_k
\end{equation}
%
%
We then have
%
\begin{align*}
f(x_{k+1}^+) \le& f(x_{k+1}) - \frac{1}{2L}\norm{\nabla f(x_{k+1})}^2&& \mbox{(Lipschitz)}\\
\le& f(x_{k+1}) - \frac{1}{2\mu}\norm{\nabla f(x_{k+1})}^2 + \frac{1}{2\mu}\left(1-\frac{1}{\kappa}\right)\norm{\nabla f(x_{k+1})}^2 && \mbox{}\\
=& \widetilde{v}  + \frac{1}{2\mu}\left(1-\frac{1}{\kappa}\right)\norm{\nabla f(x_{k+1})}^2&& \mbox{}\\
\le& v_{k+1}  + \left(1-\frac{1}{\kappa}\right)\frac{\norm{\nabla f(x_{k+1})}^2}{2\mu}&& \mbox{(QA)}\\
\le& v_{k+1}  + \frac{\kappa-1}{\kappa}\frac{\sqrt{\kappa}}{\sqrt{\kappa}+1}r_k&& \mbox{(\ref{eq:qa_proof_case1})}\\
\le& v_{k+1}  + \frac{\sqrt{\kappa}-1}{\sqrt{\kappa}}r_k= v_{k+1}  + r_{k+1}&&
\end{align*}
%
%

Now we suppose that (\ref{eq:qa_proof_case1}) is false, so
%
\begin{equation}\label{eq:qa_proof_case2}
\frac{\norm{\nabla f(x_{k+1})}^2}{\mu} \ge  2\frac{\sqrt{\kappa}}{\sqrt{\kappa}+1}r_k
\end{equation}
%

From the previous computation we have
%
\begin{align*}
\widetilde{v}  \ge  f(x_{k+1}^+) - \frac{1}{2\mu}\left(1-\frac{1}{\kappa}\right)\norm{\nabla f(x_{k+1})}^2=: \widetilde{v}_A
\end{align*}
%
We also have
%
\begin{align*}
f(x_{k+1}^+) \le& f(x_{k+1}) - \frac{1}{2L}\norm{\nabla f(x_{k+1})}^2 && \mbox{(Lipschitz)}\\
\le& f(x_{k}^+) - \frac{1}{2L}\norm{\nabla f(x_{k+1})}^2 && \mbox{(Line-search)}\\
\le&  v_{k} + r_{k} - \frac{1}{2L}\norm{\nabla f(x_{k+1})}^2&& \mbox{(Induction)}
\end{align*}
such that
\begin{align*}
v_k \ge  f(x_{k+1}^+) -r_k + \frac{1}{2\mu\kappa}\norm{\nabla f(x_{k+1})}^2=:\widehat{v}_B.
\end{align*}
%
By the line-search we have (!)
%
\begin{align*}
\scp{\nabla f(x_{k+1})}{x_{k+1}-c_k}\le 0,
\end{align*}
%
such that 
%
\begin{align*}
\norm{x_{k+1}^{++}-c_k}^2 = \norm{x_{k+1}-c_k -\frac{1}{\mu}\nabla f(x_{k+1})}^2 \ge \norm{x_{k+1}-c_k}^2 + \frac{1}{\mu^2}\norm{\nabla f(x_{k+1})}^2
\ge \frac{1}{\mu^2}\norm{\nabla f(x_{k+1})}^2
\end{align*}
%
We have
%
\begin{align*}
\abs{\widehat{v}_A-\widehat{v}_B} = \abs{r_k - \frac{1}{2\mu}\norm{\nabla f(x_{k+1})}^2},
\end{align*}
%
such that
%
\begin{align*}
\frac{\abs{\widehat{v}_A-\widehat{v}_B}}{\mu\norm{x_{k+1}^{++}-c_k}^2}\le \mu\frac{\abs{r_k - \frac{1}{2\mu}\norm{\nabla f(x_{k+1})}^2}}{\norm{\nabla f(x_{k+1})}^2} = \abs{\frac{\mu r_k}{\norm{\nabla f(x_{k+1})}^2}-\frac12}\le \frac12
\end{align*}
%
since by (\ref{eq:qa_proof_case2}) we have $0\le\frac{\mu r_k}{\norm{\nabla f(x_{k+1})}^2}\le \frac{\sqrt{\kappa}+1}{2\sqrt{\kappa}}\le1$ (and $\kappa\ge1$).

Then we have by Lemma~\ref{lemma:quadav} and $d^2:=\norm{x_{k+1}^{++}-c_k}^2$ and $h^2:=\frac{\norm{\nabla f(x_{k+1})}^2}{\mu}$
%
\begin{align*}
v_{k+1} \ge& \frac{\widehat{v}_A+\widehat{v}_B}{2}+ \frac{\mu}{8}\norm{x_{k+1}^{++}-c_k}^2+\frac{(\widehat{v}_B-\widehat{v}_A)^2}{2\mu\norm{x_{k+1}^{++}-c_k}^2}\\
v=& f(x_{k+1}^+) + \frac12\left(  \frac{h^2}{\kappa}- \frac{h^2}{2}   -r_k  \right) 
+ \frac{\mu}{8}d^2+\frac{\left( r_k - \frac{h^2}{2}\right)^2}{2\mu d^2}\\
=& f(x_{k+1}^+) -r_k  + \frac{h^2}{2\kappa}  
+\frac{\left( \frac{\mu}{2}d^2 +(r_k - \frac{h^2}{2}) \right)^2}{2\mu d^2}\\ 
=& f(x_{k+1}^+) -r_k  + \frac{h^2}{2\kappa}  
+\frac{\mu}{8}\left(d + \frac{2}{\mu}(r_k - \frac{h^2}{2})/d \right)^2 
\end{align*}
%

%
\begin{align*}
f(x_{k+1}^+) -r_k  + X \ge& f(x_{k+1}^+) -r_{k+1} = f(x_{k+1}^+) - (1-1\sqrt{\kappa})r_k\\
\quad\Leftrightarrow\quad 
\sqrt{\kappa} X \ge& r_k
\end{align*}
%




Let $\phi(s) = s + a/s$ on $[b;+\infty[$. 
Then $\phi'(s) = 1- a/s^2$, $\phi''(s) = 2a/s^3$. If $a\le 0$, $\phi$ is strictly increasing and $\phi(s)\ge \phi(b)=b+a/b$. Otherwise, $\phi$ is strictly convex with global minimum $s = \sqrt{a}$, so $\phi(s) \ge 2\sqrt{a}$ if $\sqrt{a}\ge b$. This gives with $a=\frac{2}{\mu}(r_k - \frac{h^2}{2})$ and $b=h^2/\mu$.

If $a\le 0$ we have $2\frac{\sqrt{\kappa}}{\sqrt{\kappa}+1} r \le h^2\le 2r$


If $a\le 0$ we have $h^2\ge 2r$ and $b+a/b = h^2/\mu + 2r_k/h^2 - 1$


\begin{align*}
v_{k+1} \ge&  
f(x_{k+1}^+) -r_k  + \frac{h^2}{2\kappa}  +\frac{\mu}{8}\left(h^2/\mu + 2r_k/h^2 - 1\right)^2\\
=& f(x_{k+1}^+) -r_k  +
\end{align*}
%

\blue{
%
\begin{align*}
 -r_k  + \frac{h^2}{2\kappa}  
+\frac{\mu}{4}\sqrt{\frac{2}{\mu}(r_k - \frac{h^2}{2})} \ge -r_{k+1} = -r_k(1-1/\sqrt{\kappa})\\
\quad\Leftrightarrow\quad 
\frac{h^2}{2}\frac{\mu}{L}  
+\frac{\mu}{4}\sqrt{\frac{2}{\mu}(r_k - \frac{h^2}{2})} \ge  r_k/\sqrt{\kappa} = r_k \frac{\sqrt{\mu}}{\sqrt{L}}\\
\quad\Leftrightarrow\quad 
\frac{h^2}{2}\frac{\sqrt{\mu}}{\sqrt{L}}  
+\frac{\sqrt{2L}}{4}\sqrt{(r_k - \frac{h^2}{2})} \ge  r_k\\
\end{align*}
%
}
Let $\phi(s) = as + b\sqrt{c-s}$ on $[0;c]$. Then $\phi'(s) = a - b(c-s)^{-1/2}$, $s^* = c - b^2/a^2$, 
$\phi(s^*) = ac - b^2/a + b^2+a = ac$, so
%
\begin{align*}
\frac{h^2}{2}\frac{\sqrt{\mu}}{\sqrt{L}}  
+\frac{\sqrt{2L}}{4}\sqrt{(r_k - \frac{h^2}{2})} \ge \frac{\sqrt{\mu}}{\sqrt{L}} r_k
\end{align*}
%
\end{proof}
%



%
%==========================================
\section{JGMTRT21}\label{sec:}
%==========================================
%
From \cite{JahaniGudapatiMa21}, inspired by \cite{DrusvyatskiyFazelRoy18a}.

%
%---------------------------------------
\begin{yellow}
\begin{algorithm}[H]
\caption{Accelerated Smooth Underestimate Sequence Algorithm (ASUESA)} 
\label{algorithm:QA} 
%
Inputs: $x_0\in X$, $\eps>0$. Set $k=0$, $v_0:=x_0^{++}$, $\phi^*_0:=f(x_0^+)+\left(1-\frac{1}{\kappa}\right)\frac{\norm{\nabla f(x_0)}^2}{2\mu}$, $\alpha_k = 1/\sqrt{\kappa}$, $\beta_k=1/(1+\alpha_k)=\sqrt{\kappa}/(\sqrt{\kappa}+1)$
%
\begin{itemize}
\item[(1)] $y_{k} := \beta_k x_k + (1-\beta_k)v_k$.
\item[(2)] $x_{k+1} = y_k - \frac{1}{L}\nabla f(y_k)$
\item[(3)] $v_{k+1} = (1-\alpha_k) v_k + \alpha_k y_k^{++}$,\\ 
$\phi^*_{k+1}=(1-\alpha_k)\left(\phi^*_k+\frac{\alpha_k\mu}{2}\norm{y_k^{++}-v_k}^2\right) + \alpha_k\left(f(y_k) -\frac{\norm{\nabla f(y_k)}^2}{2\mu} \right)$
\item[(4)] If $f(x_{k+1})-\phi^*_{k+1} \le \eps$: quit.
\item[(5)] Increment $k$ and go to (1).
\end{itemize}
%
\end{algorithm}
\end{yellow}
%---------------------------------------

%---------------------------------------
\begin{theorem}(Corollary 4 in \cite{JahaniGudapatiMa21})\label{thm:}
We have
%
\begin{equation}\label{eq:}
\phi^*_k \le f^* \le f(x_k^+),\quad f(x_k)-\phi^*_k \le \rho^k \left(f(x_0)-\phi^*_0\right), \quad \rho:=1-1/\sqrt{\kappa}.
\end{equation}
%
\end{theorem}
%
The idea (underestimate sequence) is to show that
%
\begin{equation}\label{eq:UES}
\phi^*_k \le f(x^*),\quad f(x_{k+1})-\phi^*_{k+1} \le (1-\alpha_k)\left(f(x_{k})-\phi^*_{k}\right) 
\end{equation}
%
which implies $f(x_{k+1})-\phi^*_{k} \le \prod_{m=0}^k(1-\alpha_k)\left(f(x_{0})-\phi^*_{0}\right)$.

The sequence is constructed recursively by
%
\begin{equation}\label{eq:UESDef}
%
\left\{
\begin{aligned}
\phi_0(x) = \phi^*_0 + \frac{\mu}{2}\norm{v_0}^2,\quad \phi_{k+1} = (1-\alpha_k)\phi_k + \alpha_k \psi(x, y_k)\\
\psi(x, y) := f(y) + \scp{\nabla f(y)}{x-y}+\frac{\mu}{2}\norm{x-y}^2
\end{aligned}
\right.
\end{equation}
%
%---------------------------------------
\begin{lemma}\label{lemma:}
We have
%
\begin{equation}\label{eq:uas_help22}
\psi(x, y) \le f(x),\quad \psi(x, y) = f(y)  + \frac{\mu}{2}\norm{x-y^{++}}^2 - \frac{\norm{\nabla f(y)}^2}{2\mu}
\end{equation}
%
\end{lemma}
%
%---------------------------------------
\begin{remark}\label{rmk:}
For composite function, the authors use instead
%
\begin{align*}
\psi(x, y) := f(y^+) + \scp{\nabla f(y)}{x-y}+\frac{\mu}{2}\norm{x-y}^2+\frac{1}{2L}\norm{\nabla f(y)}^2,
\end{align*}
%
giving
%
\begin{align*}
\psi(x, y) = f(y^+) - \left(1-\frac{1}{\kappa}\right)\frac{\norm{\nabla f(y)}^2}{2\mu} + \frac{\mu}{2}\norm{x-y^{++}}^2
\end{align*}
%


\end{remark}
%
%---------------------------------------
\begin{proof}
%
\begin{align*}
f(y)  \le f(x) - \scp{\nabla f(y)}{x-y} - \frac{\mu}{2}\norm{x-y}^2
\end{align*}
%
which gives the first assertion. With $ab+b^2/2 = (a+b)^2/2 - a^2/2$ it follows also that
%
%
\begin{align*}
\psi(x, y) =& f(y) + \scp{\nabla f(y)}{x-y}+\frac{\mu}{2}\norm{x-y}^2\\
=& f(y) + \frac{\mu}{2}\norm{x-y+\frac{1}{2\mu}\nabla f(y)}^2-\frac{1}{2\mu}\norm{\nabla f(y)}^2\\
=& f(y) + \frac{\mu}{2}\norm{x-y^{++}}^2-\frac{\norm{\nabla f(y)}^2}{2\mu}
\end{align*}
%
%
\end{proof}
%
%---------------------------------------
\begin{lemma}\label{lemma:}
We have for $(\phi_k)_{k\in\N}$ defined by (\ref{eq:UESDef})
%
\begin{equation}\label{eq:ues_recphi}
%
\left\{
\begin{aligned}
\phi_{k+1}(x) =& \phi_{k+1}^* + \frac{\mu}{2}\norm{x-v_{k+1}}^2,\quad 
v_{k+1} = (1-\alpha_k)v_k + \alpha_k  y_k^{++}\\
\phi_{k+1}^* =& (1-\alpha_k)\left(\phi_{k}^* + \frac{\alpha_k\mu}{2}\norm{v_k-y_k^{++}}^2\right) + \alpha_k\left(f(y_k) -\frac{\norm{\nabla f(y_k)}^2}{2\mu} \right).
\end{aligned}
\right.
\end{equation}
%
\end{lemma}
%
%---------------------------------------
\begin{proof}
By induction. $k=0$ is trivial.
%
\begin{align*}
\phi_{k+1}(x) =& (1-\alpha_k)\phi_k(x) + \alpha_k \psi(x, y_k) &\quad&\left(\mbox{Definition of $(\phi_k)$}\right)\\
=& (1-\alpha_k)\left(\phi_{k}^* + \frac{\mu}{2}\norm{x-v_{k}}^2\right) + \alpha_k \psi(x, y_k) &\quad&\left(\mbox{Induction}\right)\\
=& (1-\alpha_k)\left(\phi_{k}^* + \frac{\mu}{2}\norm{x-v_{k}}^2\right) + \alpha_k \left( f(y_k) - \frac{\norm{\nabla f(y_k)}^2}{2\mu} + \frac{\mu}{2}\norm{x-y_k^{++}}^2\right) &\quad&(\ref{eq:uas_help22})
\end{align*}
%
Now we have
%
%
\begin{equation}\label{eq:uas_help}
(1-\alpha)\norm{a-b}^2 + \alpha\norm{a-c}^2 = \norm{a - (1-\alpha)b-\alpha c}^2 + \alpha(1-\alpha)\norm{c-b}^2
\end{equation}
%
This is true for $\alpha=0$.
Since the right hand side is equal to
%
\begin{align*}
\norm{a - b + \alpha(b-c)}^2 + \alpha(1-\alpha)\norm{c-b}^2 
= \norm{a - b}^2 + 2\alpha\scp{a-b}{b-c}+\alpha\norm{c-b}^2 
\end{align*}
%
its derivative with respect to $\alpha$ is
%
\begin{align*}
2\scp{a-b}{b-c}+\norm{c-b}^2  = \norm{a-b}^2 - \norm{a-c}^2,
\end{align*}
%
which equals the derivative of the left hand side of (\ref{eq:uas_help}).
%
\end{proof}
%
It remains to check (\ref{eq:UES}).
%
%
We have
%
%
\begin{equation}\label{eq:ues_help33}
f(x_{k+1}) \le f(y_{k}) -  \frac{1}{2L}\norm{\nabla f(y_k)}^2 \le f(x_k) + \scp{\nabla f(y_k)}{y_k - x_k} -  \frac{1}{2L}\norm{\nabla f(y_k)}^2
\end{equation}
%
such that
%
\begin{align*}
f(x_{k+1})-\phi^*_{k+1}=& f(x_{k+1}) -(1-\alpha_k)\left(\phi_{k}^* + \frac{\alpha_k\mu}{2}\norm{v_k-y_k^{++}}^2\right) -\alpha_k\left(f(y_k) - \frac{\norm{\nabla f(y_k)}^2}{2\mu}\right)&\quad&\mbox{(\ref{eq:ues_recphi})}\\
%
=& (1-\alpha_k)\left(f(x_{k+1}) -\phi^*_k-\frac{\alpha_k\mu}{2}\norm{v_k-y_k^{++}}^2\right) + \alpha_k\left( \frac{\norm{\nabla f(y_k)}^2}{2\mu}-\frac{\norm{\nabla f(y_k)}^2}{2L}\right)&\quad&(\ref{eq:ues_help33})\\
%
\le& (1-\alpha_k)(f(x_k)-\phi_{k}^*)+ \alpha_k\left( \frac{\norm{\nabla f(y_k)}^2}{2\mu}-\frac{\norm{\nabla f(y_k)}^2}{2L}\right)-(1-\alpha_k)\frac{\norm{\nabla f(y_k)}^2}{2L}\\
&+ (1-\alpha_k)\left(\scp{\nabla f(y_k)}{y_k - x_k} -  \frac{\alpha_k\mu}{2}\norm{v_k-y_k^{++}}^2\right)
\end{align*}
By the scheme we have
%
\begin{align*}
v_k = y_k + \frac{\beta_k}{1-\beta_k}(y_k - x_k),
\end{align*}
%
so with $\frac{\beta_k}{1-\beta_k}= \frac{1}{1+\alpha_k}\frac{1+\alpha_k}{\alpha_k}=1/\alpha_k$
%
\begin{align*}
\norm{v_{k}-y_k^{++}}^2 =& \norm{\frac{\beta_k}{1-\beta_k}(y_k - x_k) + \frac{1}{\mu}\nabla f(y_k)}^2\\
= &\frac{1}{\alpha_k^2}\norm{y_k - x_k}^2 +  \frac{2}{\mu\alpha_k}\scp{y_k - x_k}{\nabla f(y_k)}
+\frac{1}{\mu^2}\norm{\nabla f(y_k)}^2
\end{align*}
%
and 
%
\begin{align*}
\scp{\nabla f(y_k)}{y_k - x_k} -  \frac{\alpha_k\mu}{2}\norm{v_k-y_k^{++}}^2 =& -\frac{\mu}{2\alpha_k}\norm{y_k - x_k}^2 -\frac{\alpha_k}{2\mu}\norm{\nabla f(y_k)}^2,
\end{align*}
%
so
%
\begin{align*}
f(x_{k+1})-\phi^*_{k+1}\le& (1-\alpha_k)(f(x_k)-\phi_{k}^*)+ \alpha_k\frac{\norm{\nabla f(y_k)}^2}{2\mu}-\frac{\norm{\nabla f(y_k)}^2}{2L}\\
&- (1-\alpha_k)\left(\frac{\mu}{2\alpha_k}\norm{y_k - x_k}^2 +\frac{\alpha_k}{2\mu}\norm{\nabla f(y_k)}^2\right)\\
=& (1-\alpha_k)(f(x_k)-\phi_{k}^*) - (1-\alpha_k)\left(\frac{\mu}{2\alpha_k}\norm{y_k - x_k}^2\right)\\
&+ \left(\frac{\alpha_k}{2\mu}-\frac{(1-\alpha_k)}{2L}- \frac{\alpha_k(1-\alpha_k)}{2\mu}\right)\norm{\nabla f(y_k)}^2
\end{align*}
%
But
%
\begin{align*}
\frac{\alpha_k}{2\mu}-\frac{1}{2L}- \frac{\alpha_k(1-\alpha_k)}{2\mu} = \frac{\alpha_k^2}{2\mu}-\frac{1}{2L}=0.
\end{align*}
%


%
%==========================================
\section{PSW21}\label{sec:}
%==========================================
%
From \cite{ParkSalgadoWise21}.

%
%---------------------------------------
\begin{yellow}
\begin{algorithm}[H]
\caption{Accelerated GM} 
\label{algorithm:PSW21AGM} 
%
Inputs: $x_0\in X$, $\eta>0$. Set $k=0$, $x_{-1}:=x_0$, $\beta= \frac{1-\eta\sqrt{s}}{1+\eta\sqrt{s}}$
%
\begin{itemize}
\item[(1)] $y_{k} :=  x_k + \beta(x_k-x_{k-1})$.
\item[(2)] $x_{k+1} = y_k - t\nabla f(y_k)$
\item[(5)] Increment $k$ and go to (1).
\end{itemize}
%
\end{algorithm}
\end{yellow}
%---------------------------------------

%---------------------------------------
\begin{lemma}\label{lemma:}
Let $\theta= \eta\sqrt{s}$ and 
%
\begin{align*}
v_{k+1} = x_k - \frac{1}{\theta} (x_{k+1} -x_{k})
\end{align*}
%

\end{lemma}
%


%-----------------------------------------------
\printbibliography
%-----------------------------------------------
%
%-------------------------------------------
\end{document}      
%-----------------------------------------{f(y^+)f(y^+)