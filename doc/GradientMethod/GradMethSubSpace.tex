%----------------------------------------
\documentclass[12pt,english]{article}
%----------------------------------------
%

%---------------------------------------------------------
% !TEX root = FemM1.tex
%----------------------------------------


%----------------------------------------
\usepackage{amsmath,amssymb,amsthm,latexsym}
%\usepackage{stmaryrd}
%\usepackage{pifont}
\usepackage{mathtools}
%\usepackage{refcheck}
\usepackage{hyperref}
%%----------------------------------------
%\usepackage[utf8]{inputenc}
\usepackage{palatino,eulervm}
%---------------------------------------------------------
\usepackage{url}
\usepackage[english,algoruled,lined]{algorithm2e}
\usepackage{listings}
\usepackage{xcolor}
\usepackage[many]{tcolorbox}
\tcbuselibrary{breakable}
\usepackage{xcolor}
\usepackage{cancel}
\usepackage{mathtools} 
\usepackage{enumitem}

% !TEX root = FemM1.tex
%----------------------------------------


%----------------------------------------

\definecolor{myyellow}{rgb}{0.9, 0.9, 0.01}
\definecolor{myblue}{rgb}{0.004, 0.1, 0.57}
\definecolor{myred}{rgb}{0.58, 0.066, 0.}
\definecolor{mygreen}{rgb}{0.24, 0.55, 0.15}
\definecolor{mygray}{rgb}{0.4, 0.6, 0.6}
\definecolor{myviolet}{rgb}{0.3, 0.1, 0.4}
\definecolor{myorange}{rgb}{0.4654205607476635, 0.33271028037383177, 0.20186915887850468}
\newcommand{\blue}[1]{\textcolor{myblue}{#1}}
\newcommand{\red}[1]{\textcolor{myred}{#1}}
\newcommand{\green}[1]{\textcolor{mygreen}{#1}}
\newcommand{\gray}[1]{\textcolor{mygray}{#1}}
\newcommand{\violet}[1]{\textcolor{myviolet}{#1}}
\definecolor{greenlight}{rgb}{0.95,1,0.95}
\definecolor{bluelight}{rgb}{0.9,0.9,1}
\definecolor{graylighy}{rgb}{0.975,0.975,0.975}
\definecolor{yellowlight}{rgb}{0.999, 0.999, 0.925}
\definecolor{roselight}{rgb}{0.99, 0.9, 0.95}
%
\newenvironment{blueenv}{\color{myblue}}{}
\newenvironment{greenenv}{\color{mygreen}}{}
\newenvironment{grayenv}{\color{mygray}}{}
\newenvironment{yellow}[1][]{\begin{tcolorbox}[breakable,title={#1} , boxrule=2pt, colback=yellowlight]}{\end{tcolorbox}}
\newenvironment{rose}[1][]{\begin{tcolorbox}[breakable,title={#1} , boxrule=2pt, colback=roselight]}{\end{tcolorbox}}



%---------------------------------------------------------
%
\newtheorem{theorem}{Theorem}
\newtheorem{lemma}{Lemma}
\newtheorem{corollary}{Corollary}
\newtheorem{proposition}{Proposition}
\newtheorem{remark}{Remark}
\newtheorem{example}{Example}
%

%----------------------------------------
\newcommand{\dpfrac}[2]{\frac{\partial #1}{\partial #2}} 
\newcommand{\ddpfracdiag}[2]{\frac{\partial^2 #1}{\partial #2^2}} 
\newcommand{\ddpfrac}[3]{\frac{\partial^2 #1}{\partial #2\partial #3}} 
\newcommand{\dddpfrac}[4]{\frac{\partial^3 #1}{\partial #2\partial #3\partial #4}} 
\newcommand{\dn}[1]{\dpfrac{#1}{n}} 
\newcommand{\dpt}[1]{\dpfrac{#1}{t}} 
\newcommand{\dptt}[1]{\frac{\partial^2 #1}{\partial t^2 }}  
\newcommand{\dbeta}[1]{\dpfrac{#1}{\beta}} 

%----------------------------------------
% OPT
%\newcommand{\argmin}{\operatorname{argmin}}
\DeclareMathOperator*{\argmin}{argmin}
\DeclareMathOperator*{\argmax}{argmax}

%---------------------------------------------------------
%
% AFEM
%
\newcommand{\REF}{{\mbox{\textbf{REF}}}}
\newcommand{\MARK}{{\mbox{\textbf{MARK}}}}
\newcommand{\allmeshes}{{\mathbb H}}
\newcommand{\meshcriterion}{\mathcal C}
%
\newcommand{\Csolve}{C_{\rm s}}
\newcommand{\Cmesh}{C_{\rm mesh}}
\newcommand{\Cclose}{C_{\rm cls}}
\newcommand{\Cref}{C_{\rm ref}}
\newcommand{\Cglobrel}{C_{\rm gr}}
\newcommand{\Cloceff}{C_{\rm le}}
\newcommand{\Copt}{C_{\rm opt}}
\newcommand{\Cneighb}{C_{\rm nei}}
\newcommand{\Cmeshopt}{C_{\rm mopt}}
\newcommand{\Cgeom}{C_{\rm g}}
\newcommand{\Cmeshcrit}{C_{\rm mc}}
\newcommand{\Cmon}{C_{\rm mon}}
\newcommand{\Cemon}{C_{\rm em}}
\newcommand{\Ceststab}{C_{\rm es}}
\newcommand{\Cinv}{C_{\rm inv}}
\newcommand{\qmesh}{q_{\rm m}}
\newcommand{\qsolve}{q_{\rm s}}

\newcommand{\chistop}[1]{{\chi_{#1}^{\rm fin}}}
\newcommand{\DeltaStop}[1]{{\widetilde{\Delta}_{#1}}}
\newcommand{\solvecrit}[1]{{\mathcal S_{#1}}}
\newcommand{\qred}{q_{\rm red}}
\newcommand{\Crel}{C_{\rm rel}}
\newcommand{\Cstab}{C_{\rm stab}}
\newcommand{\Ceff}{C_{\rm eff}}

\newcommand{\Cupp}{C_{\rm up}}
\newcommand{\Clow}{C_{\rm low}}
\newcommand{\true}{{\mathtt{true}}}
\newcommand{\false}{{\mathtt{false}}}

%---------------------------------------------------------
%
\newcommand{\Expt}[1]{{\rm E}(#1)}
\newcommand{\Var}[1]{{\rm Var}(#1)}
%
\renewcommand{\Re}{\rm Re} 
\renewcommand{\Im}{\rm Im}
\newcommand{\N}{\mathbb N}
\newcommand{\R}{\mathbb R}
\newcommand{\Z}{\mathbb Z}
\newcommand{\C}{\mathbb C}
%
% 
\newcommand{\Set}[1]{ \left\{#1\right\}}
\newcommand{\SetDef}[2]{\left\{#1\;\middle|\;#2\right\}} 
\newcommand{\Rest}[2]{{#1}_{|_{#2}}}
\newcommand{\transpose}[1]{#1^{\mathsf{T}}} 
\newcommand{\transposeInv}[1]{#1^{\mathsf{-T}}} 
%
\newcommand{\supp}[1]{\operatorname{supp}\left(#1\right)}
\renewcommand{\div}{\operatorname{div}}
\newcommand{\rot}{\operatorname{rot}}
\newcommand{\diag}{\operatorname{diag}}
\newcommand{\norm}[1]{\left\|#1\right\|}
\makeatletter
\newcommand{\opnorm}{\@ifstar\@opnorms\@opnorm}
\newcommand{\@opnorms}[1]{%
  \left|\mkern-1.5mu\left|\mkern-1.5mu\left|
   #1
  \right|\mkern-1.5mu\right|\mkern-1.5mu\right|
}
\newcommand{\@opnorm}[2][]{%
  \mathopen{#1|\mkern-1.5mu#1|\mkern-1.5mu#1|}
  #2
  \mathclose{#1|\mkern-1.5mu#1|\mkern-1.5mu#1|}
}
\makeatother
\newcommand{\abs}[1]{\left\vert{#1}\right\vert}
\newcommand{\tnorm}[1]{\opnorm{#1}}
\newcommand{\eps}{\varepsilon}
\newcommand{\scp}[2]{\langle#1,#2\rangle}
\newcommand{\Pe}{\mathrm {Pe}}
\newcommand{\Sides}{\mathcal S}
\newcommand{\Cells}{\mathcal K}
\newcommand{\Nodes}{\mathcal N}
\newcommand{\NodesInt}{{\mathcal N}^{int}}
\newcommand{\level}{\operatorname{lev}}
\newcommand{\SidesInt}{\mathcal S^{\rm int}}
\newcommand{\SidesBdry}{\mathcal S^{\partial}}
\newcommand{\Kin}{K^{\rm{\footnotesize{in}}}}
\newcommand{\Kex}{K^{\rm{\footnotesize{ex}}}}
\newcommand{\inS}[1]{{#1}^{\rm{\footnotesize{in}}}_{S}}
\newcommand{\exS}[1]{{#1}^{\rm{\footnotesize{ex}}}_{S}}
\newcommand{\inSs}[1]{{#1}^{\rm{\footnotesize{in}}}}
\newcommand{\exSs}[1]{{#1}^{\rm{\footnotesize{ex}}}}
\newcommand{\meanS}[1]{\left\{\{#1\right\}_{S}}
\newcommand{\jumpS}[1]{\left[#1\right]_{S}}
\newcommand{\jump}[1]{\left[#1\right]}
\newcommand{\mean}[1]{\left\{#1\right\}}
\newcommand{\intS}{\int_{\Sides_{h}}}
\newcommand{\intSInt}{\int_{\SidesInt_{h}}}
\newcommand{\intSBound}{\int_{\SidesBound_{h}}}
%\newcommand{\intK}{\int_{\Cells_{h}}}
\newcommand{\REFLOC}{{\rm REF}}
\newcommand{\AFEM}[1]{\textbf{AFEM}($#1$)}

%
\newcommand{\tesfctu}{\delta\! u}
\newcommand{\udir}{u^{\rm D}}
\newcommand{\vdir}{v^{\rm D}}
\newcommand{\wdir}{w^{\rm D}}
\newcommand{\uh}{u_{h}}
\newcommand{\phih}{\phi_{h}}
\newcommand{\uin}{u^{\rm in}}
\newcommand{\uex}{u^{\rm ex}}
\newcommand{\uhin}{u^{\rm in}_{h}}
\newcommand{\uhex}{u^{\rm ex}_{h}}
\newcommand{\ubar}{\overline{u}}
\newcommand{\ustar}{u^{*}}
\newcommand{\sgn}[1]{\operatorname{sgn}(#1)}
%
\newcommand{\In}[1]{#1^{\rm in}}
\newcommand{\Ex}[1]{#1^{\rm ex}}
\newcommand{\InDe}[2]{#1^{{\rm in}_{#2}}}
\newcommand{\ExDe}[2]{#1^{{\rm ex}_{#2}}}

%---------------------------------------------------------

\newcommand{\GS}{\mbox{\boldmath\textbf{GSp}}}


%-------------------------------------------
\begin{document}
%-------------------------------------------

\title{Optimization algorithms with approximation}
\author{}
\maketitle
\tableofcontents

%\begin{abstract}
%\end{abstract}
%\linenumbers
%
%==========================================
\section{Introduction}\label{sec:}
%==========================================
%
%
We consider a Hilbert space $(X,\scp{\cdot}{\cdot})$ with induced norm $\norm{\cdot}$ and the minimization of a smooth $\mu$-strictly convex function $f:X\to\R$:
%
\begin{align*}
\inf_{x\in X} f(x) = \inf\SetDef{f(x)}{x\in X}.
\end{align*}
%
We suppose that a unique minimizer $x^*$ exists.
%

Our purpose is to analyse gradient algorithms on a sequence of subspaces (finite element spaces for the PDE)
%
\begin{align*}
X_0 \subset \cdots \subset X_k \subset X_{k+1}\subset \cdots \subset X,\quad P_k: X\to X_k,
\end{align*}
%
such that a typical iteration reads: 
%
\begin{yellow}
\begin{equation}\label{eq:pgm}
x_{k+1} = x_k - t_k P_{k}\nabla f(x_k),
\end{equation}
\end{yellow}
%
where $P_k$ is the orthogonal projector on $X_k$ and $\nabla f(x)\in X$ is defined by the Riesz map. In order to generate the subspaces $X_k$, we suppose to have 
an error estimators $\eta_k:X_k\to\R$ and a refinement algorithm satisfying typical hypothesis from the theory of AFEM.
% 
In the case $f\in \mathcal S^{1,1}_{\mu,L}(X)$, $X$ finite-dimensional, and $t_k=2/(\mu+L)$ for all $k$ we have the following {convergence estimate}  (Theorem 2.1.15 in \cite{Nesterov18}) for the gradient method (\textbf{GM}):
%
\begin{equation}\label{eq:gmlipsmoothest}
\norm{x_n-x^*} \le \rho^n\norm{x_0-x^*},\quad \rho = 1-1/\kappa,
\end{equation}
%
such that $\eps>0$ is achieved in $n(\eps) = O(\kappa)\ln(1/\eps)$ iterations.
It is well-known that \textbf{GM} is not optimal for this class of functions. The accelerated gradient method (\textbf{AGM}) \cite{Nesterov18} yields an improved estimate  $n(\eps) = O(\sqrt{\kappa})\ln(1/\eps)$.

Our aim is to establish a similar iteration count for the method on a sequence of subspaces.
There is important progress of adaptive finite element methods (AFEM) for nonlinear elliptic equations, 
see \cite{ErnVohralik13a,HeidWihler20,HeidPraetoriusWihler21,HaberlPraetoriusSchimanko21,GantnerHaberlPraetorius21,HeidWihler21,HeidStammWihler21,HeidWihler22}, and our development is based on these works. However, here, we wish to 
work out the optimization point of view.
%
%\cite{}
%==========================================
\section{Notation}\label{sec:}
%==========================================
%
We throughout suppose that $f:X\to\R$ is convex and $C^1$ and we use the the Fréchet-Riesz theorem to define
%
\begin{align*}
\scp{\nabla f(y)}{x} = f'(y)(x)\quad \forall x,y\in X.
\end{align*}
%
It is then easy to see, that for all $y$ in a closed subspace $Y\subset X$ and $P_Y:X\to Y$ its orthogonal projector
%
\begin{equation}\label{eq:gradrest}
P_Y\nabla f(y) = \nabla \Rest{f}{Y}.
\end{equation}
%

%==========================================
\section{Assumptions on subspace selection}\label{sec:}
%==========================================
%

Let $X_0\subset X$ be a subspace. We suppose to have a lattice of admissible closed subspaces 
%
\begin{equation}\label{eq:}
\mathcal X(X_0) = \Set{X_0\subset Y\subset X}.
\end{equation}
%
The partial order on $\mathcal X(X_0)$ is given by $Y\ge Z$ if and only if $Y$ is a superspace of $Z$. We 
then have the finest common coarsening $Y\land Z$ and the coarsest common refinement $Y\lor Z$, respectively.
We let
%
\begin{align*}
\mathcal X(Y) = \SetDef{Z\in \mathcal X(X_0)}{Y\land Z = Y},\quad Y\in \mathcal X(x_0).
\end{align*}
%
We make the following assumptions. First we have a reliable error estimator $\eta$ such that for all $Y\in \mathcal X(X_0)$
%
\begin{align}
\label{eq:hyp:estimator:reliability}\tag{H1}
\norm{(I-P_Y)\nabla f(y)}\le& \Crel \eta(y,Y)&\quad&\forall y\in Y\\
\label{hyp:estimator:stability}\tag{H2}
\abs{\eta(y, Y)-\eta(z, Y)}\le& \Cstab \norm{y-z}&\quad&\forall y,z\in Y
\end{align}
%
and a subspace generator $Y^+=\GS(Y, y)$ such that with $0\le\qred<1$
%
\begin{align}
\label{hyp:estimator:reduction}\tag{H2}
%
\eta^2(y, Y^+)\le& \qred \eta^2(y,Y) &\quad&\forall y\in Y
%
\end{align}
%


%==========================================
\section{Gradient method with constant step-size}\label{sec:}
%==========================================
%
Here we suppose in addition that $f$ has a $L$-Lipschitz continuous gradient
%
\begin{align*}
\norm{\nabla f(x)-\nabla f(y)} \le L \norm{x-y}\quad \forall x,y\in X.
\end{align*}
%
Setting $\beta=0$ in the following algorithm, we have the standard gradient method with fixed step-size.
%
%---------------------------------------
\begin{yellow}
\begin{algorithm}[H]
\caption{GM with constant step-size} 
\label{algorithm:Descent} 
%
Inputs: $X_0, x_0\in X_0$, $t_0>0, 0\le\beta<1$, $\lambda>0$. Set $x_{-1}=x_0$ and $k=0$.
%
\begin{itemize}
\item[(1)] $y_{k} = (1+\beta)x_k - \beta x_{k-1}$.
\item[(2)] $x_{k+1} = y_k - \frac{1}{L}P_{X_k}\nabla f(y_k)$.
\item[(3)] If $\eta(x_{k+1},X_k) \ge \qred \eta(x_{k},X_k) + \lambda(f(x_k)-f(x_{k+1})$:\\ 
$\qquad X_{k+1} = \GS(X_K, x_{k+1})$,\\
Else: $X_{k+1} = X_k$.
\item[(4)] Increment $k$ and go to (1).
\end{itemize}
%
\end{algorithm}
\end{yellow}
%---------------------------------------

We will write for brevity $P_k = P_{X_k}$ etc.
By the Lipschitz-condition and convexity we have for any $x\in X$ with (\ref{eq:gradrest})
%
\begin{align*}
f(x_{k+1}) \le& f(y_k) + \scp{\nabla f(y_k)}{x_{k+1}-y_k} + \frac{L}{2}\norm{P_k\nabla f(y_k)}^2\\
=&  f(y_k) + \scp{P_k \nabla f(y_k)}{x_{k+1}-y_k} + \frac{L}{2}\norm{P_k\nabla f(y_k)}^2\\
\le& f(x) + \scp{\nabla f(y_k)}{y_k-x}-\frac{\mu}{2}\norm{x-y_k}^2 - \frac{1}{2L}\norm{P_k\nabla f(y_k)}^2
\end{align*}
%
Let $\theta = \frac{1-\beta}{1+\beta}$. Taking the last inequality $\theta$-times with $x=x^*$ and 
$1-\theta$-times with $x=x_k$, we have, setting $\Delta f_k:=f(x_k)-f(x^*)$
%
\begin{align*}
\Delta f_{k+1} - (1-\theta)\Delta f_k \le& \scp{\nabla f(y_k)}{y_k-(1-\theta) x_k -\theta x^*} - \frac{1}{2L}\norm{P_k\nabla f(y_k)}^2\\&-\frac{\mu\theta}{2}\norm{x^*-y_k}^2-\frac{\mu(1-\theta)}{2}\norm{x_k-y_k}^2
\end{align*}
%
Let
%
\begin{align*}
R_k := \theta\scp{(I-P_k)\nabla f(y_k)}{y_k-x^*},
\end{align*}
%
such that
%
\begin{align*}
\scp{\nabla f(y_k)}{y_k-(1-\theta) x_k -\theta x^*} = R_k + \scp{P_k\nabla f(y_k)}{y_k-(1-\theta) x_k -\theta x^*}
\end{align*}
%

and
%
\begin{align*}
z_k := \frac{x_k}{\theta} -\frac{1-\theta}{\theta}x_{k-1}= x_k +\frac{1-\theta}{\theta}(x_k-x_{k-1}).
\end{align*}
%
We also have with $\theta(1+\beta) = 1-\beta$
%
\begin{align*}
z_k = y_k  + \frac{1-\theta-\theta\beta}{\theta\beta}(y_k-x_{k})= y_k  + \frac{y_k-x_{k}}{\theta}
\end{align*}
%

Then with $2ab-a^2=b^2-(a-b)^2$
%
\begin{align*}
&\scp{P_k\nabla f(y_k)}{y_k-(1-\theta) x_k -\theta x^*}-\frac{1}{2L}\norm{P_k\nabla f(y_k)}^2=\\
&\frac{L}{2}\left(\norm{y_k-(1-\theta) x_k -\theta x^*}^2-\norm{x_{k+1}-(1-\theta) x_k -\theta x^*}^2\right)=\\
&\frac{\theta^2L}{2}\left(\norm{x_k+\frac{y_k-x_k}{\theta} - x^*}^2-\norm{z_{k+1}- x^*}^2\right)=
\frac{\theta^2L}{2}\left(\norm{z_k -(y_k-x_k) - x^*}^2-\norm{z_{k+1}- x^*}^2\right)
\end{align*}
%
Since with $-2ab= (a-b)^2-a^2-b^2$
%
\begin{align*}
\norm{z_k -(y_k-x_k) - x^*}^2 =& \norm{z_k - x^*}^2 - 2\theta\scp{z_k -x^*}{z_k-y_k} + \norm{y_k-x_k}^2\\
=& (1-\theta) \norm{z_k - x^*}^2 + (\theta^2-\theta)\norm{z_k-y_k}^2 + \theta \norm{y_k-x^*}^2
\end{align*}
%
we have
%
%
\begin{align*}
\Delta f_{k+1} + (1-\theta)\Delta f_k \le& \frac{\theta^2L}{2}\left((1-\theta)\norm{z_k - x^*}^2-\norm{z_{k+1}- x^*}^2\right)
+ \left(\frac{\theta^3L}{2}-\frac{\theta\mu}{2}\right)\norm{y_k-x^*}^2 + R_k
\end{align*}
%
We have
%
\begin{align*}
(1-\theta)\norm{z_k - x^*}^2 + \theta \norm{y_k-x^*}^2 =& \norm{(1-\theta)z_k+\theta y_k - x^*}^2 + \theta(1-\theta)\norm{z_k -y_k}^2\\
=& \norm{y_k  + (1-\theta)\frac{y_k-x_{k}}{\theta} - x^*}^2 + \frac{1-\theta}{\theta}\norm{y_k -x_k}^2
\end{align*}
%

%
\begin{align*}
R_k \le \frac{1}{\mu}\norm{(I-P_k)\nabla f(y_k)}^2 + \frac{\theta\mu}{4}\norm{y_k-x^*}^2
\le  \frac{\Crel^2}{\mu}\eta^2(X_k,y_k)  + \frac{\theta\mu}{4}\norm{y_k-x^*}^2
\end{align*}
%
If the criterion in step (3) of the algorithm does not hold, we have
%
\begin{align*}
\eta(x_{k+1},X_k) \le \qred \eta(x_{k},X_k) + \lambda(f(x_k)-f(x_{k+1})
\end{align*}
%
Otherwise, we have
%
\begin{align*}
\eta(x_{k+1},X_k) \le \qred \eta(x_{k+1},X_k)\le \qred \eta(x_{k},X_k)+ \Cstab\norm{x_{k+1}-x_{k}}
\end{align*}
%

%




%-----------------------------------------------
\printbibliography
%-----------------------------------------------
%
%-------------------------------------------
\end{document}      
%-------------------------------------------