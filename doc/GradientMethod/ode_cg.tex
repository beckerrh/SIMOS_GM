% !TEX encoding = UTF-8 Unicode
%----------------------------------------
\documentclass[english,12pt,a4paper]{article}
%----------------------------------------
%
%----------------------------------------
% !TEX root = FemM1.tex
%----------------------------------------


%----------------------------------------
\usepackage{amsmath,amssymb,amsthm,latexsym}
%\usepackage{stmaryrd}
%\usepackage{pifont}
\usepackage{mathtools}
%\usepackage{refcheck}
\usepackage{hyperref}
%%----------------------------------------
%\usepackage[utf8]{inputenc}
\usepackage{palatino,eulervm}
%---------------------------------------------------------
\usepackage{url}
\usepackage[english,algoruled,lined]{algorithm2e}
\usepackage{listings}
\usepackage{xcolor}
\usepackage[many]{tcolorbox}
\tcbuselibrary{breakable}
\usepackage{xcolor}
\usepackage{cancel}
\usepackage{mathtools} 
\usepackage{enumitem}

% !TEX root = FemM1.tex
%----------------------------------------


%----------------------------------------

\definecolor{myyellow}{rgb}{0.9, 0.9, 0.01}
\definecolor{myblue}{rgb}{0.004, 0.1, 0.57}
\definecolor{myred}{rgb}{0.58, 0.066, 0.}
\definecolor{mygreen}{rgb}{0.24, 0.55, 0.15}
\definecolor{mygray}{rgb}{0.4, 0.6, 0.6}
\definecolor{myviolet}{rgb}{0.3, 0.1, 0.4}
\definecolor{myorange}{rgb}{0.4654205607476635, 0.33271028037383177, 0.20186915887850468}
\newcommand{\blue}[1]{\textcolor{myblue}{#1}}
\newcommand{\red}[1]{\textcolor{myred}{#1}}
\newcommand{\green}[1]{\textcolor{mygreen}{#1}}
\newcommand{\gray}[1]{\textcolor{mygray}{#1}}
\newcommand{\violet}[1]{\textcolor{myviolet}{#1}}
\definecolor{greenlight}{rgb}{0.95,1,0.95}
\definecolor{bluelight}{rgb}{0.9,0.9,1}
\definecolor{graylighy}{rgb}{0.975,0.975,0.975}
\definecolor{yellowlight}{rgb}{0.999, 0.999, 0.925}
\definecolor{roselight}{rgb}{0.99, 0.9, 0.95}
%
\newenvironment{blueenv}{\color{myblue}}{}
\newenvironment{greenenv}{\color{mygreen}}{}
\newenvironment{grayenv}{\color{mygray}}{}
\newenvironment{yellow}[1][]{\begin{tcolorbox}[breakable,title={#1} , boxrule=2pt, colback=yellowlight]}{\end{tcolorbox}}
\newenvironment{rose}[1][]{\begin{tcolorbox}[breakable,title={#1} , boxrule=2pt, colback=roselight]}{\end{tcolorbox}}



%---------------------------------------------------------
%
\newtheorem{theorem}{Theorem}
\newtheorem{lemma}{Lemma}
\newtheorem{corollary}{Corollary}
\newtheorem{proposition}{Proposition}
\newtheorem{remark}{Remark}
\newtheorem{example}{Example}
%

%----------------------------------------
\newcommand{\dpfrac}[2]{\frac{\partial #1}{\partial #2}} 
\newcommand{\ddpfracdiag}[2]{\frac{\partial^2 #1}{\partial #2^2}} 
\newcommand{\ddpfrac}[3]{\frac{\partial^2 #1}{\partial #2\partial #3}} 
\newcommand{\dddpfrac}[4]{\frac{\partial^3 #1}{\partial #2\partial #3\partial #4}} 
\newcommand{\dn}[1]{\dpfrac{#1}{n}} 
\newcommand{\dpt}[1]{\dpfrac{#1}{t}} 
\newcommand{\dptt}[1]{\frac{\partial^2 #1}{\partial t^2 }}  
\newcommand{\dbeta}[1]{\dpfrac{#1}{\beta}} 

%----------------------------------------
% OPT
%\newcommand{\argmin}{\operatorname{argmin}}
\DeclareMathOperator*{\argmin}{argmin}
\DeclareMathOperator*{\argmax}{argmax}

%---------------------------------------------------------
%
% AFEM
%
\newcommand{\REF}{{\mbox{\textbf{REF}}}}
\newcommand{\MARK}{{\mbox{\textbf{MARK}}}}
\newcommand{\allmeshes}{{\mathbb H}}
\newcommand{\meshcriterion}{\mathcal C}
%
\newcommand{\Csolve}{C_{\rm s}}
\newcommand{\Cmesh}{C_{\rm mesh}}
\newcommand{\Cclose}{C_{\rm cls}}
\newcommand{\Cref}{C_{\rm ref}}
\newcommand{\Cglobrel}{C_{\rm gr}}
\newcommand{\Cloceff}{C_{\rm le}}
\newcommand{\Copt}{C_{\rm opt}}
\newcommand{\Cneighb}{C_{\rm nei}}
\newcommand{\Cmeshopt}{C_{\rm mopt}}
\newcommand{\Cgeom}{C_{\rm g}}
\newcommand{\Cmeshcrit}{C_{\rm mc}}
\newcommand{\Cmon}{C_{\rm mon}}
\newcommand{\Cemon}{C_{\rm em}}
\newcommand{\Ceststab}{C_{\rm es}}
\newcommand{\Cinv}{C_{\rm inv}}
\newcommand{\qmesh}{q_{\rm m}}
\newcommand{\qsolve}{q_{\rm s}}

\newcommand{\chistop}[1]{{\chi_{#1}^{\rm fin}}}
\newcommand{\DeltaStop}[1]{{\widetilde{\Delta}_{#1}}}
\newcommand{\solvecrit}[1]{{\mathcal S_{#1}}}
\newcommand{\qred}{q_{\rm red}}
\newcommand{\Crel}{C_{\rm rel}}
\newcommand{\Cstab}{C_{\rm stab}}
\newcommand{\Ceff}{C_{\rm eff}}

\newcommand{\Cupp}{C_{\rm up}}
\newcommand{\Clow}{C_{\rm low}}
\newcommand{\true}{{\mathtt{true}}}
\newcommand{\false}{{\mathtt{false}}}

%---------------------------------------------------------
%
\newcommand{\Expt}[1]{{\rm E}(#1)}
\newcommand{\Var}[1]{{\rm Var}(#1)}
%
\renewcommand{\Re}{\rm Re} 
\renewcommand{\Im}{\rm Im}
\newcommand{\N}{\mathbb N}
\newcommand{\R}{\mathbb R}
\newcommand{\Z}{\mathbb Z}
\newcommand{\C}{\mathbb C}
%
% 
\newcommand{\Set}[1]{ \left\{#1\right\}}
\newcommand{\SetDef}[2]{\left\{#1\;\middle|\;#2\right\}} 
\newcommand{\Rest}[2]{{#1}_{|_{#2}}}
\newcommand{\transpose}[1]{#1^{\mathsf{T}}} 
\newcommand{\transposeInv}[1]{#1^{\mathsf{-T}}} 
%
\newcommand{\supp}[1]{\operatorname{supp}\left(#1\right)}
\renewcommand{\div}{\operatorname{div}}
\newcommand{\rot}{\operatorname{rot}}
\newcommand{\diag}{\operatorname{diag}}
\newcommand{\norm}[1]{\left\|#1\right\|}
\makeatletter
\newcommand{\opnorm}{\@ifstar\@opnorms\@opnorm}
\newcommand{\@opnorms}[1]{%
  \left|\mkern-1.5mu\left|\mkern-1.5mu\left|
   #1
  \right|\mkern-1.5mu\right|\mkern-1.5mu\right|
}
\newcommand{\@opnorm}[2][]{%
  \mathopen{#1|\mkern-1.5mu#1|\mkern-1.5mu#1|}
  #2
  \mathclose{#1|\mkern-1.5mu#1|\mkern-1.5mu#1|}
}
\makeatother
\newcommand{\abs}[1]{\left\vert{#1}\right\vert}
\newcommand{\tnorm}[1]{\opnorm{#1}}
\newcommand{\eps}{\varepsilon}
\newcommand{\scp}[2]{\langle#1,#2\rangle}
\newcommand{\Pe}{\mathrm {Pe}}
\newcommand{\Sides}{\mathcal S}
\newcommand{\Cells}{\mathcal K}
\newcommand{\Nodes}{\mathcal N}
\newcommand{\NodesInt}{{\mathcal N}^{int}}
\newcommand{\level}{\operatorname{lev}}
\newcommand{\SidesInt}{\mathcal S^{\rm int}}
\newcommand{\SidesBdry}{\mathcal S^{\partial}}
\newcommand{\Kin}{K^{\rm{\footnotesize{in}}}}
\newcommand{\Kex}{K^{\rm{\footnotesize{ex}}}}
\newcommand{\inS}[1]{{#1}^{\rm{\footnotesize{in}}}_{S}}
\newcommand{\exS}[1]{{#1}^{\rm{\footnotesize{ex}}}_{S}}
\newcommand{\inSs}[1]{{#1}^{\rm{\footnotesize{in}}}}
\newcommand{\exSs}[1]{{#1}^{\rm{\footnotesize{ex}}}}
\newcommand{\meanS}[1]{\left\{\{#1\right\}_{S}}
\newcommand{\jumpS}[1]{\left[#1\right]_{S}}
\newcommand{\jump}[1]{\left[#1\right]}
\newcommand{\mean}[1]{\left\{#1\right\}}
\newcommand{\intS}{\int_{\Sides_{h}}}
\newcommand{\intSInt}{\int_{\SidesInt_{h}}}
\newcommand{\intSBound}{\int_{\SidesBound_{h}}}
%\newcommand{\intK}{\int_{\Cells_{h}}}
\newcommand{\REFLOC}{{\rm REF}}
\newcommand{\AFEM}[1]{\textbf{AFEM}($#1$)}

%
\newcommand{\tesfctu}{\delta\! u}
\newcommand{\udir}{u^{\rm D}}
\newcommand{\vdir}{v^{\rm D}}
\newcommand{\wdir}{w^{\rm D}}
\newcommand{\uh}{u_{h}}
\newcommand{\phih}{\phi_{h}}
\newcommand{\uin}{u^{\rm in}}
\newcommand{\uex}{u^{\rm ex}}
\newcommand{\uhin}{u^{\rm in}_{h}}
\newcommand{\uhex}{u^{\rm ex}_{h}}
\newcommand{\ubar}{\overline{u}}
\newcommand{\ustar}{u^{*}}
\newcommand{\sgn}[1]{\operatorname{sgn}(#1)}
%
\newcommand{\In}[1]{#1^{\rm in}}
\newcommand{\Ex}[1]{#1^{\rm ex}}
\newcommand{\InDe}[2]{#1^{{\rm in}_{#2}}}
\newcommand{\ExDe}[2]{#1^{{\rm ex}_{#2}}}

%====================================================

\title{FEM for ODEs}
\author{Roland Becker}
\date{\today}
\usepackage{environ}



%====================================================
\begin{document}
%====================================================
\maketitle
\setcounter{tocdepth}{2}
\tableofcontents
%
%==========================================
\section{Introduction}\label{sec:}
%==========================================
%
%
We consider the smooth $n$-dimesniosnal ODE
%
\begin{equation}\label{eq:ode}
u'(t) = f(u(t)) + l(t),\quad t\in I=]0,T[,\qquad u(0) = u_0.
\end{equation}
%
With $U:= H^1(I,\R^n)$ and $V:=L^2(I,\R^n)$ a weak formulation is
%
\begin{equation}\label{eq:ode_weak_primal}
u\in U:\; \int_0^T \scp{u' - f(u)}{v} + \scp{u(0)}{v_0} = \int_0^T \scp{l}{v} + \scp{u_0}{v_0} \quad\forall (v,v_0)\in V\times\R^n.
\end{equation}
%
Using the relation $\int_{0}^T u'v = - \int_{0}^T uv' + uv\vert_0^T$
An alternative formulation is given by: Find $u,u_T\in V\times\R^n$, such that
%
\begin{equation}\label{eq:ode_weak_dual}
-\int_0^T \left(\scp{u}{v'}+ \scp{f(u)}{v}\right)+ \scp{u_T}{v(T)} = \int_0^T \scp{l}{v} + \scp{u_0}{v(0)} \quad\forall v\in U.
\end{equation}
%
%
%==========================================
\section{FEM discretization}\label{sec:}
%==========================================
%
%
We let $\delta=(0=t_0<t_1<\cdots < t_N=T)$ be a partition, $I_{\ell}:=]t_{\ell-1},t_{\ell}[$, $1\le \ell\le N$, $\delta_{\ell}:=\abs{I_{\ell}}$. 


We let $U_{\delta}\subset U$ and $V_{\delta}\subset V$ be two conforming piecewise polynomial spaces and consider the semi-implicit discretization: Find $u_{\delta}\in U_{\delta}$ such that 
for all $(v,v_0)\in V_{\delta}\times\R^n$
%
\begin{equation}\label{eq:ode_weak_delta}
\int_0^T \scp{u_{\delta}' - \left( f(\tilde u) + f'_u(\tilde u)(u_{\delta}-\tilde u)\right)}{v} + \scp{u_{\delta}(0)}{v_0} = \int_0^T \scp{l}{v} + \scp{u_0}{v_0}.
\end{equation}
%
where
%
\begin{equation}\label{eq:}
\Rest{\tilde u}{I_{\ell}} := u_{\ell-1},\qquad u_{\ell-1} := u(t_{\ell-1}).
\end{equation}
%
This gives on each time interval the linear system of equations
%
\begin{align*}
\int_{I_{\ell}} \scp{u_{\delta}' - A u_{\delta})}{v}  = \int_{I_{\ell}} \scp{l+f(u_{\ell-1})-A u_{\ell-1}}{v} \quad \forall v\in P^{k-1}(I_{\ell}).
\end{align*}
%
with $A:=f'_u(u_{\ell-1})$.  Now we suppose that 
%
\begin{equation}\label{eq:ode_weak_u_in_basis}
\Rest{u}{I_{\ell}} = u_{\ell-1} + \sum_{j=1}^k c_j \phi_j(t),\quad \phi_j(0)= 0,\quad \deg\phi_j = j,\; 1\le j\le k-1. 
\end{equation}
%
Then on each time interval we have to solve
%
\begin{align*}
\int_{I_{\ell}} \scp{\phi_j' -  A\phi_j}{v}c_j  = \int_{I_{\ell}} \scp{l+ f(u_0)}{v} \quad \forall v\in P^{k-1}(I_{\ell}).
\end{align*}
%
Let $l_i$  be the Legendre functions normalized by $\int_0^1 l_il_j=\delta_{ij}$, $0\le i,j$. 

Taking as basis for $P^{k-1}(I)$  $\psi_i = l_{i-1}$, $1\le i\le k$ and 
$\phi_j(t) = \int_0^t l_{j-1}(s)\,ds$ for $1\le j\le k-1$ we have for $j\le k-1$
%
\begin{align*}
M_{ij} = \int_0^1 \phi_j'l_i = \int_0^1 l_j l_i
\end{align*}
%
%
%-------------------------------------------------------------------------
\subsection{Dual scheme}\label{subsec:}
%-------------------------------------------------------------------------
%
In the same way, the weak formulation (\ref{eq:ode_weak_dua}) leads to the dual scheme: Find $u_{\delta},u_T\in V_{\delta}\times\R^n$, such that
%
\begin{equation}\label{eq:ode_weak_dual_delta}
-\int_0^T \left(\scp{u_{\delta}}{v'}+ \scp{f(u_{\delta})}{v}\right)+ \scp{u_T}{v(T)} = \int_0^T \scp{l}{v} + \scp{u_0}{v(0)} \quad\forall v\in U_{\delta}.
\end{equation}
%
If we use linearization, we have
%
\begin{equation}\label{eq:ode_weak_dual_delta_lin}
-\int_0^T \left(\scp{u_{\delta}}{v'}+ \scp{A u_{\delta}}{v}\right)+ \scp{u_T}{v(T)} = \int_0^T \scp{l+f(\tilde u) - A \tilde u}{v} + \scp{u_0}{v(0)} \quad\forall v\in U_{\delta}.
\end{equation}
%

%
%==========================================
\section{Abstract setting}\label{sec:}
%==========================================
%
In order to put (\ref{eq:weak_form}) and (\ref{eq:weak_form_ap}) in conforming the Babuska-framework   \cite{XuZikatanov03}
%
\begin{equation}\label{eq:}
%
\left\{
\begin{aligned}
X_{\delta}\times Y_{\delta} \subset& X\times Y\\
x\in X:\; a(x)(y) =& b(y)\quad\forall y\in Y,\\ 
x_{\delta}\in X_{\delta}:\; a_{\delta}(x_{\delta})(y) =& b(y)\quad\forall y\in Y_{\delta}.
\end{aligned}
\right.
%
\end{equation}
%
We  let
%
\begin{equation}\label{eq:}
\left\{
\begin{aligned}
X := H^1(I,\R^n),\quad \norm{x}_X := \left(\norm{x'}_{L^2(I,\R^n)}^2 + \norm{x(0)}^2\right)^{\frac12}\\
Y := L^2(I,\R^n) \times \R^n,\quad \norm{(y_1,y_0)}_Y := \left( \norm{y_1}_{L^2(I,\R^n)}^2 + \norm{y_0}^2 \right)^{\frac12}
\end{aligned}
\right.
\end{equation}
%
%
Let us suppose the continuous inf-sup uniform condition
%
\begin{equation}\label{eq:abstract_infsup}
\gamma := \inf_{x\in X\setminus\Set{0}}\sup_{y\in Y\setminus\Set{0}} \frac{a'(x_0)(x,y)}{\norm{x}_X\norm{y}_Y}  > 0,\quad \forall x_0\in X.
\end{equation}
%
and its discrete version
%
\begin{equation}\label{eq:abstract_infsup_delta}
\gamma_{\delta} := \inf_{x\in X_{\delta}\setminus\Set{0}}\sup_{y\in Y_{\delta}\setminus\Set{0}} \frac{a_{\delta}'(x_0)(x,y)}{\norm{x}_X\norm{y}_Y}  > 0,\quad \forall x_0\in X_{\delta}.
\end{equation}
%
%~~~~~~~~~~~~~~~~~~~~~~~~~~~~~
\subsubsection{A priori}
%~~~~~~~~~~~~~~~~~~~~~~~~~~~~~
%
Let $\widetilde{x}_{\delta} \in X_{\delta}$
%
%
%~~~~~~~~~~~~~~~~~~~~~~~~~~~~~
\subsubsection{A posteriori}
%~~~~~~~~~~~~~~~~~~~~~~~~~~~~~
%
%---------------------------------------
\begin{theorem}\label{thm:}
We have
%
\begin{equation}\label{eq:}
%
\left\{
\begin{aligned}
\gamma \norm{x-x_{\delta}}_X \le& R_1 + R_2,\\
 R_1 :=& \sup_{y\in Y\setminus\Set{0}}\inf_{y_{\delta}\in Y_{\delta}\setminus\Set{0}}\frac{b(y-y_{\delta}) - a_{\delta}(x_{\delta})(y-y_{\delta})}{\norm{y}_Y} ,\\ 
 R_2 :=&  \sup_{y\in Y\setminus\Set{0}}\frac{a(x_{\delta})(y)-a_{\delta}(x_{\delta})(y)}{\norm{y}_Y}.
\end{aligned}
\right.
%
\end{equation}
%
\end{theorem}
%
%---------------------------------------
\begin{proof}
We have for any $y\in Y$ and $y_{\delta}\in Y_{\delta}$
%
\begin{align*}
&\int_0^1a'(x_{\delta}+t(x-x_{\delta}))(x-x_{\delta},y)\,dt = a(x)(y) - a(x_{\delta})(y)
= b(y) - a(x_{\delta})(y)\\ 
&= b(y) - a_{\delta}(x_{\delta})(y) + a(x_{\delta})(y)-a_{\delta}(x_{\delta})(y)\\
&= b(y-y_{\delta}) - a_{\delta}(x_{\delta})(y-y_{\delta}) + a(x_{\delta})(y)-a_{\delta}(x_{\delta})(y) \le \left(R_1+R_2\right)\norm{y}_{Y}
\end{align*}
%
%
Then
%
\begin{align*}
\gamma \norm{x-x_{\delta}} \le& \int_0^1\sup_{y\in Y\setminus\Set{0}}\frac{a'(x_{\delta}+t(x-x_{\delta}))(x-x_{\delta},y)}{\norm{y}_Y}\,dt\\ =& 
\sup_{y\in Y\setminus\Set{0}}\frac{a(x)(y) - a(x_{\delta})(y)}{\norm{y}_Y} \le R_1 + R_2.
\end{align*}
%
\end{proof}
%
%
%==========================================
\section{A posterior error estimator}\label{sec:}
%==========================================
%
%
%-------------------------------------------------------------------------
\subsection{Primal scheme}\label{subsec:}
%-------------------------------------------------------------------------
%
%
We have for $y=(v,v_0)\in L^2(I,\R^n) \times \R^n$ and $y_{\delta}=(v_{\delta},v_0)$ with $v_{\delta} = \pi_{\delta}v$. Since $u_{\delta}'\in V_{\delta}$ and $f(\widetilde{u_{\delta}}) + f'(\widetilde{u_{\delta}})(\widetilde{u_{\delta}})\in V_{\delta}$ we have
%
\begin{align*}
b(y-y_{\delta}) - a_{\delta}(u_{\delta})(y-y_{\delta}) =  \int_{0}^T\scp{l  + f'(\widetilde{u_{\delta}})(u_{\delta})}{v-v_{\delta}}
\end{align*}
%
so
%
\begin{align*}
R_1 \le \sum_{\ell=1}^N \eta^1_{\ell}(u_{\delta})\norm{v}_{L^2(I_{\ell})} \le \left(\sum_{\ell=1}^N \eta^1_{\ell}(u_{\delta})^2\right)^{\frac12}\norm{v}_{L^2(I)} 
\end{align*}
%
With
%
\begin{align*}
 \eta^1_{\ell}(u_{\delta}) := \norm{(I-\pi_{\delta})\left( l + f'(u_{\ell-1})u_{\delta}\right)}_{L^2(I_{\ell})}
\end{align*}
%
%---------------------------------------
\begin{lemma}\label{lemma:}
Let $u$ have the development as in (\ref{eq:ode_weak_u_in_basis}). Then
%
\begin{equation}\label{eq:}
\norm{(I-\pi_{\delta})f'(u_{\ell-1})u_{\delta}}_{I_{\ell}} = \norm{f'(u_{\ell-1})c_k}^2 \norm{\phi_{k}}_{L^2(I_{\ell})}
\end{equation}
% 
\end{lemma}
%
%---------------------------------------
\begin{proof}
%
\end{proof}
%
%
Similarly we have for $R_2$
%
\begin{align*}
a(x_{\delta})(y)-a_{\delta}(x_{\delta})(y) = \sum_{k=1}^{N}\int_{I_{\ell}}\scp{f(u_{\delta}) - f(\widetilde{u_{\delta}})}{v}
\end{align*}
%
%
\begin{align*}
f(u_{\delta}) - f(\widetilde{u_{\delta}}) =& f(u_{\delta}) - f(u_{\ell-1}) - f'(u_{\ell-1})(u_{\delta}-u_{\ell-1})\\
=& \int_0^1\left(f'((1-s)u_{\ell-1} + s u_{\delta})- f'(u_{\ell-1})\right)\,ds(u_{\delta}-u_{\ell-1})
\end{align*}
%
If $f'$ is quadratic, the simpson rule gives for the integral
%
\begin{align*}
\frac23\left( f'(\frac{u_{\ell-1} + u_{\ell}}{2})- f'(u_{\ell-1})\right) + \frac16\left( f'(u_{\ell})- f'(u_{\ell-1})\right)
\end{align*}
%
For trapez we get
%
\begin{align*}
\frac12\left( f'(u_{\ell})- f'(u_{\ell-1})\right)
\end{align*}
%
%
%---------------------------------------
\begin{lemma}\label{lemma:}

%
\begin{equation}\label{eq:}
\eta^2_{\ell}(u_{\delta}) = \frac12\norm{f'(u_{\ell})- f'(u_{\ell-1})}\norm{u_\delta - u_{\ell-1}}_{L^2(I_{\ell})}
\end{equation}
% 
\end{lemma}
%
%
%-------------------------------------------------------------------------
\subsection{Dual scheme}\label{subsec:}
%-------------------------------------------------------------------------
%
We have to consider with $v\in H^1(I,X)$ and $w=v-v_{\delta}$
%%
%
\begin{align*}
-\int_0^T \left(\scp{u_{\delta}}{w'}+ \scp{A u_{\delta}}{w}\right)+ \scp{u_T}{w(T)} = \int_0^T \scp{l+f(\tilde u) - A \tilde u}{w} + \scp{u_0}{w(0)} \quad\forall v\in U_{\delta}.
\end{align*}
%
Integration by parts gives
%
\begin{align*}
&-\int_0^T \scp{u_{\delta}}{w'} + \scp{u_T}{w(T)} - \scp{u_0}{w(0)}= \int_0^T \scp{u_{\delta}'}{w} \\
&- \sum_{\ell=1}^{N-1} \scp{[u_{\delta}(t_{\ell})]}{w(t_{\ell})} + \scp{u_{\delta}(0)-u0}{w(0)} - \scp{u_{\delta}(T)-u_T}{w(T)}
\end{align*}
%



%
%==========================================
\section{Analysis in the linear case $f(u)=-Au$ with SPD $A$}\label{sec:}
%==========================================
%
The equation reads
%
\begin{equation}\label{eq:ode_A}
u\in U:\; \int_0^T \scp{u' +Au}{v} + \scp{u(0)}{v_0} = \int_0^T \scp{l}{v} + \scp{u_0}{v_0} \quad\forall (v,v_0)\in V\times\R^n.
\end{equation}
%
%
We suppose $A$ to symmetric and positive definite and denote $\norm{u}_{A^k}=\norm{A^{k/2}u}$, $k\in\Z$. We equip $U$ and $V$ with the norms
%
\begin{align*}
\norm{u}_U^2 := \norm{A^{-\frac12}u'}^2_{L^2(I,\R^n)} + \norm{A^{\frac12}u}^2_{L^2(I,\R^n)} + \norm{u(0)}^2_{\R^n} + \norm{u(T)}^2_{\R^n},\quad 
\norm{v}_V^2 := \norm{A^{\frac12}v}^2_{L^2(I,\R^n)}
\end{align*}
%
Denoting the bilinear form on the left of (\ref{eq:ode_A}) by $a$, we wish to show that
%
\begin{equation}\label{eq:ode_A_infsup}
\inf_{u\in U\setminus\Set{0}}\inf_{(v,v_0)\in V\times\R^n\setminus\Set{0}} \frac{a(u,v)}{\norm{u}_U\norm{v}_V}=\gamma>0.
\end{equation}
%
First, testing with $(v,v_0)=(A^{-1}(u'+ Au),2u(0))$ we have
%
\begin{align*}
\norm{(v,v_0)}^2_{V\times\R^n} \le 4\norm{u(0)}^2_{\R^n} + \norm{A^{-\frac12}(u'+ Au)}^2_{L^2(I,\R^n)} \le 4\norm{u}^2_U.
\end{align*}
%
and
%
\begin{align*}
a(u, (v,v_0)) =& \norm{A^{-\frac12}u'}^2_{L^2(I,\R^n)} + 2\norm{u(0)}^2 + 2\int_0^T \scp{u}{u'} + \norm{A^{\frac12}u}^2_{L^2(I,\R^n)}\\
=& \norm{A^{-\frac12}u'}^2_{L^2(I,\R^n)} + \norm{u(1)}^2 + \norm{u(0)}^2  + \norm{A^{\frac12}u}^2_{L^2(I,\R^n)}
\end{align*}
%
since
%
\begin{align*}
 2\int_0^T \scp{u}{u'} = \norm{u(1)}^2 - \norm{u(0)}^2.
\end{align*}
%
This yields $\gamma\ge\frac12$ in (\ref{eq:ode_A_infsup}).



%
For the discrete scheme
%
\begin{equation}\label{eq:ode_A_delta}
u_{\delta}\in U_{\delta}:\; \int_0^T \scp{u_{\delta}' +Au_{\delta}}{v} + \scp{u(0)}{v_0} = \int_0^T \scp{l}{v} + \scp{u_0}{v_0} \quad\forall (v,v_0)\in V_{\delta}\times\R^n.
\end{equation}
%
we let $\pi_{\delta}:L^2(I,\R^n)\to V_{\delta}$ be the $L^2(I,\R^n)$ projection. Then, testing with $(v,v_0)=(A^{-1}\pi_{\delta}(u_{\delta}'+ Au_{\delta}),2u_{\delta}(0))$ yields
%
\begin{align*}
a(u_{\delta}, (v,v_0)) =& \norm{A^{-\frac12}\pi_{\delta}u_{\delta}'}^2_{L^2(I,\R^n)} + 2\norm{u_{\delta}(0)}^2 + 2\int_0^T \scp{\pi_{\delta}u_{\delta}}{u_{\delta}'} + \norm{A^{\frac12}\pi_{\delta}u_{\delta}}^2_{L^2(I,\R^n)}
\end{align*}
%
In case
%
\begin{equation}\label{eq:ode_A_delta_condCN}
u_{\delta}' \in V_{\delta}
\end{equation}
%
we get 
%
\begin{align*}
a(u_{\delta}, (v,v_0)) = \norm{A^{-\frac12}u_{\delta}'}^2_{L^2(I,\R^n)} + \norm{u_{\delta}(1)}^2 + \norm{u_{\delta}(0)}^2  + \norm{A^{\frac12}\pi_{\delta}u_{\delta}}^2_{L^2(I,\R^n)}
\end{align*}
%
which is weaker, since it only controls the projection of the solution in the $A$-norm. This explains the oscillations of the Crank-Nicolson scheme for the heat equation.


Without (\ref{eq:ode_A_delta_condCN}) we have
%
\begin{align*}
a(u_{\delta}, (v,v_0)) =& \norm{A^{-\frac12}\pi_{\delta}u_{\delta}'}^2_{L^2(I,\R^n)} + \norm{u_{\delta}(1)}^2 + \norm{u_{\delta}(0)}^2\\
& + 2\int_0^T \scp{\pi_{\delta}u_{\delta}-u_{\delta}}{u_{\delta}'} + \norm{A^{\frac12}\pi_{\delta}u_{\delta}}^2_{L^2(I,\R^n)}
\end{align*}
%
We get the correct norms under the assumptions
%
%
\begin{equation}\label{eq:ode_A_delta_stab1}
\norm{A^{-\frac12}u_{\delta}'}^2_{L^2(I,\R^n)} \lesssim \norm{A^{-\frac12}\pi_{\delta}u_{\delta}'}^2_{L^2(I,\R^n)}
\end{equation}
%
%
and
%
\begin{equation}\label{eq:ode_A_delta_stab1}
\norm{A^{\frac12}\left(\pi_{\delta}u_{\delta}- u_{\delta}\right)}^2_{L^2(I,\R^n)} \lesssim \int_0^T \scp{\pi_{\delta}u_{\delta}-u_{\delta}}{u_{\delta}'} + \norm{A^{\frac12}\pi_{\delta}u_{\delta}}^2_{L^2(I,\R^n)}
\end{equation}
%


%
%
%==========================================
\section{Linearization (semi-implicit scheme)}\label{sec:}
%==========================================
%

by $D^k_{\delta}$ and $P^k_{\delta}$ the spaces of general and continuous piecewise $k$-th order polynomials, respectively. We note that
$\dim D^k_{\delta} = (N-1)(k+1)$ and $\dim P^k_{\delta} = N + (N-1)(k-1) = \dim D^{k-1}_{\delta}+1$. 
%
Let $f_{\delta}$ be a piecewise polynomial approximation of $f$. We define
%
\begin{equation}\label{eq:}
a_{\delta}(u)(v,v_0) = \sum_{k=0}^{N-1}\int_{I_{\ell}}\scp{u'(t) - f_k^{\delta}(u)}{v(t)}\, dt + \scp{u(0)}{v_0}
\end{equation}
%
and the discrete problem for $k\in \N$ 
%
\begin{equation}\label{eq:weak_form_ap}
u_{\delta}\in P^k_{\delta}:\; a(u_{\delta})(v,v_0) = b(v,v_0)\quad \forall (v,v_0)\in D^{k-1}_{\delta}\times \R^n.
\end{equation}
%
The choice, with $u_k:=u(t_{\ell})$,
%
\begin{equation}\label{eq:}
f_k^{\delta}(u) = f(u_k) +  f'(u_k)(u-u_k)
\end{equation}
%
leads to a semi-implicit scheme.

%
%==========================================
\section{Definition of the method}\label{sec:}
%==========================================
%
We consider the smooth autonomous ODE
%
\begin{equation}\label{eq:ode}
u'(t) = f(u(t)),\quad t\in I=]0,T[,\qquad u(0) = u_0.
\end{equation}
%
We let $\delta=(0=t_0<t_1<\cdots < t_N=T)$ be a partition, $I_{\ell}:=]t_{\ell-1},t_k[$, $1\le k\le N$. We denote by $D^k_{\delta}$ and $P^k_{\delta}$ the spaces of general and continuous piecewise $k$-th order polynomials. We note that
$\dim D^k_{\delta} = (N-1)(k+1)$ and $\dim P^k_{\delta} = N + (N-1)(k-1) = \dim D^{k-1}_{\delta}+1$. We define the function spaces $X=H^1(I,\R^n)$ and $Y=L^2(I,\R^n)\times \R^n$ and the form
$a:X\times Y\to \R$
%
\begin{equation}\label{eq:ode_form}
a(u)(v,w) := \int_I(u'(t) - f(u(t)))v(t)\, dt + \scp{u(0)}{w}.
\end{equation}
%
Then with the linear form
%
\begin{equation}\label{eq:}
b(v,w) := \scp{u_0}{w}
\end{equation}
%
a weak formulation of (\ref{eq:ode}) reads
%
\begin{align*}
u\in X:\; a(u)(v,w) = b(v,w)\quad \forall (v,w)\in Y.
\end{align*}
%
Let $f_{\delta}$ be a piecewise polynomial approximation of $f$. We define for $k\in \N$ $X_{\delta}:=P^k_{\delta}$, $Y_{\delta}:=D^{k-1}_{\delta}\times \R^n$,
%
\begin{equation}\label{eq:}
a_{\delta}(u)(v) = \sum_{\ell=1}^N\int_{I_{\ell}}(u'(t) - f_{\delta}(u))v(t)\, dt + \scp{u(0)}{w}
\end{equation}
%
and the discrete problem
%
\begin{align*}
u_{\delta}\in X_{\delta}:\; a(u_{\delta})(v,w) = b(v,w)\quad \forall (v,w)\in Y_{\delta}.
\end{align*}
%
%
%==========================================
\section{CG2 variants with linearization}\label{sec:}
%==========================================
%
We use a quadratic approximation written be means of an hierarchical basis with piecewise linear test functions and linearization
%
\begin{equation}\label{eq:}
f_{\delta}(u) = f(u_0) + f'(u_0)(u-u_0).
\end{equation}
% 
Transforming all intervals to $[0,1]$ we have the development 
%
\begin{equation}\label{eq:cg2_quadratic_onrefint}
u(t) = (1-t)u_0 + t u_1 + t(1-t)u_2
\end{equation}
%
with $u_0$ known and $u_1$ and $u_2$ verifying
%
\begin{align*}
\int_0^1 \left( u'(t) - (f(u_0) + f'(u_0)(u-u_0))\right) \psi(t)\,dt = 0\quad \psi \in\Psi.\\
\int_0^1 \left( (u_1-u_0) + (1-2t)u_2 - (f(u_0) + f'(u_0)( t (u_1-u_0) + t(1-t)u_2 )\right) \psi(t)\,dt = 0\quad \psi \in\Psi.\\
\int_0^1 \left( u_1 + (1-2t)u_2 - f'(u_0)( t u_1 + t(1-t)u_2 ) \right) \psi(t)\,dt = \int_0^1 \left( u_0 + f(u_0) - t f'(u_0)u_0  )\right) \psi(t)\,dt\quad \psi \in\Psi.\\
\int_0^1 \left( u_1 + (1-2t)u_2 - f'(u_0)( t u_1 + t(1-t)u_2 ) \right) \psi(t)\,dt = \int_0^1 \left( u_0 + f(u_0) - t f'(u_0)u_0  )\right) \psi(t)\,dt\quad \psi \in\Psi.
\end{align*}
%
%
Denoting $A:=f'(u_0)$ and be $b:=u_0 + f(u_0)$ we have
%
\begin{align*}
\alpha(\psi) u_1 + \beta(\psi) u_2= \int_0^1  b  \psi(t)\,dt + \alpha(\psi) u_0,\quad \psi \in \Psi\\
\alpha(\psi):= \int_0^1 \left( M - tA \right) \psi(t)\,dt,\quad \beta(\psi) := \int_0^1 \left( (1-2t)M - t(1-t)A \right) \psi(t)\,dt
\end{align*}
%
%
%-------------------------------------------------------------------------
\subsection{CG2-DG1}\label{subsec:}
%-------------------------------------------------------------------------
%
With $\Psi=\Set{1, 1-t}$ we have
%
\begin{align*}
\int_0^1t(1-t)=\frac16,\quad \int_0^1 (1-2t)(1-t)=\frac16,\quad \int_0^1 t(1-t)^2=\frac12 - \frac23 + \frac14=\frac1{12}
\end{align*}
%
%
\begin{align*}
\begin{bmatrix}
M - \frac12 A & \frac16 A\\
\frac12 M - \frac16 A &  \frac16 M - \frac1{12} A
\end{bmatrix}
\begin{bmatrix}
u_1 \\ u_2
\end{bmatrix}
=
\begin{bmatrix}
b \\\frac12 b
\end{bmatrix}
+
\begin{bmatrix}
(M - \frac12 A) u_0  \\ (\frac12 M - \frac16 A) u_0 
\end{bmatrix}
\end{align*}
%
%
%-------------------------------------------------------------------------
\subsection{CG2-2DG0}\label{subsec:}
%-------------------------------------------------------------------------
%
With $\Psi=\Set{\chi_{[0,1]}, \chi_{[0,\frac12]}}$ we have
%
\begin{align*}
\int_0^{\frac12} t\,dt = \int_0^{\frac12} (1-2t)\,dt = \frac14,\quad \int_0^{\frac12} t(1-t)\,dt = \frac12(\frac23\times\frac14\times\frac34 + \frac16\times\frac14 )=\frac{1}{12},\quad
\int_0^{\frac12} (1-t)\,dt =\frac14
\end{align*}
%

%
\begin{align*}
\begin{bmatrix}
M - \frac12 A & \frac16 A\\
\frac12 M - \frac14 A & \frac14 M - \frac1{12} A
\end{bmatrix}
\begin{bmatrix}
u_1 \\ u_2
\end{bmatrix}
=
\begin{bmatrix}
b \\\frac12 b
\end{bmatrix}
+
\begin{bmatrix}
(M - \frac12 A) u_0  \\ (\frac12 M - \frac14 A) u_0 
\end{bmatrix}
\end{align*}
%
%
%-------------------------------------------------------------------------
\subsection{2CG1-DG1}\label{subsec:}
%-------------------------------------------------------------------------
%
We replace the quadratic in (\ref{eq:cg2_quadratic_onrefint}) by a piecewise linear
%
\begin{equation}\label{eq:cg2_quadratic_onrefint}
u(t) = (1-t)u_0 + t u_1 + \phi(t) u_2,\quad \phi(t) = \frac12\min\Set{t, 1-t} = \frac{1-|2t-1|}{4} 
\end{equation}
%
Then with
%
\begin{align*}
\int_0^1 \phi(t) = \frac18,\quad \int_0^1 \phi(t)(1-t) = \frac{1}{16},\quad \int_0^1 \phi'(t) = 0,\quad \int_0^1 \phi'(t)(1-t) =\frac12\frac34\frac12 - \frac12\frac12\frac14=\frac18
\end{align*}
%
With $\Psi=\Set{1, 1-t}$ we have
%
\begin{align*}
\begin{bmatrix}
M - \frac12 A & \frac18 A\\
\frac12 M - \frac16 A &  \frac18 M - \frac{1}{16}A
\end{bmatrix}
\begin{bmatrix}
u_1 \\ u_2
\end{bmatrix}
=
\begin{bmatrix}
b \\\frac12 b
\end{bmatrix}
+
\begin{bmatrix}
(M - \frac12 A) u_0  \\ (\frac12 M - \frac14 A) u_0 
\end{bmatrix}
\end{align*}
%
%
%-------------------------------------------------------------------------
\subsection{2CG1-2DG0}\label{subsec:}
%-------------------------------------------------------------------------
%
We replace the quadratic in (\ref{eq:cg2_quadratic_onrefint}) by a piecewise linear
%
\begin{equation}\label{eq:cg2_quadratic_onrefint}
u(t) = (1-t)u_0 + t u_1 + \phi(t) u_2,\quad \phi(t) = \frac12\min\Set{t, 1-t} = \frac{1-|2t-1|}{4} 
\end{equation}
%
Then with $\Psi=\Set{\chi_{[0,1]}, \chi_{[0,\frac12]}}$ and
%
\begin{align*}
\int_0^1 \phi(t) = \frac18,\quad \int_0^{\frac12} \phi(t) = \frac{1}{16},\quad \int_0^1 \phi'(t) = 0,\quad \int_0^{\frac12} \phi'(t) = \frac14
\end{align*}
%
%
\begin{align*}
\begin{bmatrix}
M - \frac12 A & \frac18 A\\
\frac12 M - \frac18 A &  \frac14 M - \frac{1}{16}A
\end{bmatrix}
\begin{bmatrix}
u_1 \\ u_2
\end{bmatrix}
=
\begin{bmatrix}
b \\\frac12 b
\end{bmatrix}
+
\begin{bmatrix}
(M - \frac12 A) u_0  \\ (\frac12 M - \frac14 A) u_0 
\end{bmatrix}
\end{align*}
%
%
%-------------------------------------------------------------------------
\subsection{CG2$^+$-DG0}\label{subsec:}
%-------------------------------------------------------------------------
%
Another variant is to force the quadratic approximation to be $C_1$. One needs careful scaling on variable intervals and we have to decide what to do on the first interval.
On $[t_n, t_{n+1}]$, $t_{n+1}= t_n+\delta_n$ we have
%
\begin{align*}
\Rest{u}{I_n}(t) = \frac{t_{n+1}-t}{\delta_n}u_0 + \frac{t-t_n}{\delta_n} u_1 + \frac{(t_{n+1}-t)(t-t_n)}{\delta_n^2} u_2 \\\quad\Rightarrow\quad 
{\Rest{u}{I_n}}'(t_n) = \frac{u_1-u_0 + u_2}{\delta_n} 
\end{align*}
%
Denoting the previous values by $u_{-1}$ and $u_{-2}$, i.e on $[t_{n-1}, t_n]$ we have 
%
\begin{align*}
\Rest{u}{I_{n-1}}(t) = \frac{t_n-t}{\delta_{n-1}}u_{-1} + \frac{t-t_{n-1}}{\delta_{n-1}} u_0 + \frac{(t_{n}-t)(t-t_{n-1})}{\delta_{n-1}^2} u_2 \\\quad\Rightarrow\quad 
{\Rest{u}{I_{n-1}}}'(t_n) = \frac{u_0-u_{-1}-u_{-2}}{\delta_{n-1}} 
\end{align*}
%
So the $C^1$-condition reads
%
\begin{align*}
 u_1  + u_2 = \left(1+\frac{\delta_{n-1}}{\delta_n}\right)u_0 -\frac{\delta_{n-1}}{\delta_n}\left( u_{-1}+u_{-2}\right).
\end{align*}
%
%
\begin{align*}
\begin{bmatrix}
M - \frac12 A & \frac16 A\\
M & M
\end{bmatrix}
\begin{bmatrix}
u_1 \\ u_2
\end{bmatrix}
=
\begin{bmatrix}
(M + \frac12 A) u_0 + b \\ 2 u_0 - u_{-1} - u_{-2}
\end{bmatrix}
\end{align*}
%
%==========================================
\section{General analysis}\label{sec:}
%==========================================
%
We consider 
%
\begin{equation}\label{eq:}
%
\left\{
\begin{aligned}
x\in X:\quad a(x)(y) =& l(y)\quad \forall y \in Y\\
x_{\delta}\in X_{\delta}:\quad a(x_{\delta})(y) =& l(y)\quad \forall y \in Y_{\delta}
\end{aligned}
\right.
\end{equation}
%
Let us suppose the continuous inf-sup uniform condition
%
\begin{equation}\label{eq:abstract_infsup}
\gamma := \inf_{x\in X\setminus\Set{0}}\sup_{y\in Y\setminus\Set{0}} \frac{a'(x_0)(x,y)}{\norm{x}_X\norm{y}_Y}  > 0,\quad \forall x_0\in X.
\end{equation}
%
and its discrete version
%
\begin{equation}\label{eq:abstract_infsup_delta}
\gamma_{\delta} := \inf_{x\in X_{\delta}\setminus\Set{0}}\sup_{y\in Y_{\delta}\setminus\Set{0}} \frac{a_{\delta}'(x_0)(x,y)}{\norm{x}_X\norm{y}_Y}  > 0,\quad \forall x_0\in X_{\delta}.
\end{equation}
%
%
%-------------------------------------------------------------------------
\subsection{A priori}\label{subsec:}
%-------------------------------------------------------------------------
%
Let $\widetilde{x}_{\delta} \in X_{\delta}$

%
%-------------------------------------------------------------------------
\subsection{A posteriori}\label{subsec:}
%-------------------------------------------------------------------------
%
We have for any $y\in Y$
%
\begin{align*}
\int_0^1a'(x_{\delta}+t(x-x_{\delta}))(x-x_{\delta},y)\,dt =& a(x)(y) - a(x_{\delta})(y)
= l(y) - a(x_{\delta})(y)\\ \le&  \left(R_1 + R_2\right)\norm{y}_Y,
\end{align*}
%
with
%
\begin{align*}
R_1 := \sup_{y\in Y\setminus\Set{0}}\inf_{y_{\delta}\in Y_{\delta}\setminus\Set{0}}\frac{l(y-y_{\delta}) - a_{\delta}(x_{\delta})(y-y_{\delta})}{\norm{y}_Y} ,\quad R_2 :=  \sup_{y\in Y\setminus\Set{0}}\frac{a(x_{\delta})(y)-a_{\delta}(x_{\delta})(y)}{\norm{y}_Y}. 
\end{align*}
%
Then
%
\begin{align*}
\gamma \norm{x-x_{\delta}} \le& \int_0^1\sup_{y\in Y\setminus\Set{0}}\frac{a'(x_{\delta}+t(x-x_{\delta}))(x-x_{\delta},y)}{\norm{y}_Y}\,dt\\ =& 
\sup_{y\in Y\setminus\Set{0}}\frac{a(x)(y) - a(x_{\delta})(y)}{\norm{y}_Y} \le R_1 + R_2.
\end{align*}
%







%====================================================
%====================================================
\end{document}
%===================================================w