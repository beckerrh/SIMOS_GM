%----------------------------------------
\documentclass[12pt,english]{article}
%----------------------------------------
%
\usepackage{amsmath,amssymb,amsthm,latexsym}
\usepackage[a4paper,margin=1.8cm]{geometry}
%\usepackage{stmaryrd}
%\usepackage{pifont}
\usepackage{mathtools}
%\usepackage{refcheck}
\usepackage{hyperref}
%%----------------------------------------
%\usepackage[utf8]{inputenc}
\usepackage{palatino,eulervm}
\usepackage[backend=bibtex,sorting=none,giveninits=true, style=numeric, doi=false,isbn=false,url=false,maxbibnames=9]{biblatex}
\bibliography{../../../../Latex/Bibliotheque/bibliotheque.bib}
%---------------------------------------------------------
\usepackage{url}
\usepackage[english,algoruled,lined]{algorithm2e}
\usepackage{listings}
\usepackage{xcolor}
\usepackage[many]{tcolorbox}
\tcbuselibrary{breakable}
%\usepackage[toc,page]{appendix}


%\frenchbsetup{StandardLists=true}
\renewcommand{\thefigure}{\thesection.\arabic{figure}}
\newcommand*\circled[1]{\tikz[baseline=(char.base)]{
  \node[shape=circle,draw,inner sep=1pt] (char) {#1};}}
\newenvironment{rappel}{\begin{tcolorbox}[boxrule=1pt, colback=gray!10!white]}{\end{tcolorbox}}

%---------------------------------------------------------
% !TEX root = FemM1.tex
%----------------------------------------


%----------------------------------------

\definecolor{myyellow}{rgb}{0.9, 0.9, 0.01}
\definecolor{myblue}{rgb}{0.004, 0.1, 0.57}
\definecolor{myred}{rgb}{0.58, 0.066, 0.}
\definecolor{mygreen}{rgb}{0.24, 0.55, 0.15}
\definecolor{mygray}{rgb}{0.4, 0.6, 0.6}
\definecolor{myviolet}{rgb}{0.3, 0.1, 0.4}
\definecolor{myorange}{rgb}{0.4654205607476635, 0.33271028037383177, 0.20186915887850468}
\newcommand{\blue}[1]{\textcolor{myblue}{#1}}
\newcommand{\red}[1]{\textcolor{myred}{#1}}
\newcommand{\green}[1]{\textcolor{mygreen}{#1}}
\newcommand{\gray}[1]{\textcolor{mygray}{#1}}
\newcommand{\violet}[1]{\textcolor{myviolet}{#1}}
\definecolor{greenlight}{rgb}{0.95,1,0.95}
\definecolor{bluelight}{rgb}{0.9,0.9,1}
\definecolor{graylighy}{rgb}{0.975,0.975,0.975}
\definecolor{yellowlight}{rgb}{0.999, 0.999, 0.925}
\definecolor{roselight}{rgb}{0.99, 0.9, 0.95}
%
\newenvironment{blueenv}{\color{myblue}}{}
\newenvironment{greenenv}{\color{mygreen}}{}
\newenvironment{grayenv}{\color{mygray}}{}
\newenvironment{yellow}[1][]{\begin{tcolorbox}[breakable,title={#1} , boxrule=2pt, colback=yellowlight]}{\end{tcolorbox}}
\newenvironment{rose}[1][]{\begin{tcolorbox}[breakable,title={#1} , boxrule=2pt, colback=roselight]}{\end{tcolorbox}}



%---------------------------------------------------------
%
\newtheorem{theorem}{Theorem}
\newtheorem{lemma}{Lemma}
\newtheorem{corollary}{Corollary}
\newtheorem{proposition}{Proposition}
\newtheorem{remark}{Remark}
\newtheorem{example}{Example}
%

%----------------------------------------
\newcommand{\dpfrac}[2]{\frac{\partial #1}{\partial #2}} 
\newcommand{\ddpfracdiag}[2]{\frac{\partial^2 #1}{\partial #2^2}} 
\newcommand{\ddpfrac}[3]{\frac{\partial^2 #1}{\partial #2\partial #3}} 
\newcommand{\dddpfrac}[4]{\frac{\partial^3 #1}{\partial #2\partial #3\partial #4}} 
\newcommand{\dn}[1]{\dpfrac{#1}{n}} 
\newcommand{\dpt}[1]{\dpfrac{#1}{t}} 
\newcommand{\dptt}[1]{\frac{\partial^2 #1}{\partial t^2 }}  
\newcommand{\dbeta}[1]{\dpfrac{#1}{\beta}} 

%----------------------------------------
% OPT
%\newcommand{\argmin}{\operatorname{argmin}}
\DeclareMathOperator*{\argmin}{argmin}
\DeclareMathOperator*{\argmax}{argmax}

%---------------------------------------------------------
%
% AFEM
%
\newcommand{\REF}{{\mbox{\textbf{REF}}}}
\newcommand{\MARK}{{\mbox{\textbf{MARK}}}}
\newcommand{\allmeshes}{{\mathbb H}}
\newcommand{\meshcriterion}{\mathcal C}
%
\newcommand{\Csolve}{C_{\rm s}}
\newcommand{\Cmesh}{C_{\rm mesh}}
\newcommand{\Cclose}{C_{\rm cls}}
\newcommand{\Cref}{C_{\rm ref}}
\newcommand{\Cglobrel}{C_{\rm gr}}
\newcommand{\Cloceff}{C_{\rm le}}
\newcommand{\Copt}{C_{\rm opt}}
\newcommand{\Cneighb}{C_{\rm nei}}
\newcommand{\Cmeshopt}{C_{\rm mopt}}
\newcommand{\Cgeom}{C_{\rm g}}
\newcommand{\Cmeshcrit}{C_{\rm mc}}
\newcommand{\Cmon}{C_{\rm mon}}
\newcommand{\Cemon}{C_{\rm em}}
\newcommand{\Ceststab}{C_{\rm es}}
\newcommand{\Cinv}{C_{\rm inv}}
\newcommand{\qmesh}{q_{\rm m}}
\newcommand{\qsolve}{q_{\rm s}}

\newcommand{\chistop}[1]{{\chi_{#1}^{\rm fin}}}
\newcommand{\DeltaStop}[1]{{\widetilde{\Delta}_{#1}}}
\newcommand{\solvecrit}[1]{{\mathcal S_{#1}}}
\newcommand{\qred}{q_{\rm red}}
\newcommand{\Crel}{C_{\rm rel}}
\newcommand{\Cstab}{C_{\rm stab}}
\newcommand{\Ceff}{C_{\rm eff}}

\newcommand{\Cupp}{C_{\rm up}}
\newcommand{\Clow}{C_{\rm low}}
\newcommand{\true}{{\mathtt{true}}}
\newcommand{\false}{{\mathtt{false}}}

%---------------------------------------------------------
%
\newcommand{\Expt}[1]{{\rm E}(#1)}
\newcommand{\Var}[1]{{\rm Var}(#1)}
%
\renewcommand{\Re}{\rm Re} 
\renewcommand{\Im}{\rm Im}
\newcommand{\N}{\mathbb N}
\newcommand{\R}{\mathbb R}
\newcommand{\Z}{\mathbb Z}
\newcommand{\C}{\mathbb C}
%
% 
\newcommand{\Set}[1]{ \left\{#1\right\}}
\newcommand{\SetDef}[2]{\left\{#1\;\middle|\;#2\right\}} 
\newcommand{\Rest}[2]{{#1}_{|_{#2}}}
\newcommand{\transpose}[1]{#1^{\mathsf{T}}} 
\newcommand{\transposeInv}[1]{#1^{\mathsf{-T}}} 
%
\newcommand{\supp}[1]{\operatorname{supp}\left(#1\right)}
\renewcommand{\div}{\operatorname{div}}
\newcommand{\rot}{\operatorname{rot}}
\newcommand{\diag}{\operatorname{diag}}
\newcommand{\norm}[1]{\left\|#1\right\|}
\makeatletter
\newcommand{\opnorm}{\@ifstar\@opnorms\@opnorm}
\newcommand{\@opnorms}[1]{%
  \left|\mkern-1.5mu\left|\mkern-1.5mu\left|
   #1
  \right|\mkern-1.5mu\right|\mkern-1.5mu\right|
}
\newcommand{\@opnorm}[2][]{%
  \mathopen{#1|\mkern-1.5mu#1|\mkern-1.5mu#1|}
  #2
  \mathclose{#1|\mkern-1.5mu#1|\mkern-1.5mu#1|}
}
\makeatother
\newcommand{\abs}[1]{\left\vert{#1}\right\vert}
\newcommand{\tnorm}[1]{\opnorm{#1}}
\newcommand{\eps}{\varepsilon}
\newcommand{\scp}[2]{\langle#1,#2\rangle}
\newcommand{\Pe}{\mathrm {Pe}}
\newcommand{\Sides}{\mathcal S}
\newcommand{\Cells}{\mathcal K}
\newcommand{\Nodes}{\mathcal N}
\newcommand{\NodesInt}{{\mathcal N}^{int}}
\newcommand{\level}{\operatorname{lev}}
\newcommand{\SidesInt}{\mathcal S^{\rm int}}
\newcommand{\SidesBdry}{\mathcal S^{\partial}}
\newcommand{\Kin}{K^{\rm{\footnotesize{in}}}}
\newcommand{\Kex}{K^{\rm{\footnotesize{ex}}}}
\newcommand{\inS}[1]{{#1}^{\rm{\footnotesize{in}}}_{S}}
\newcommand{\exS}[1]{{#1}^{\rm{\footnotesize{ex}}}_{S}}
\newcommand{\inSs}[1]{{#1}^{\rm{\footnotesize{in}}}}
\newcommand{\exSs}[1]{{#1}^{\rm{\footnotesize{ex}}}}
\newcommand{\meanS}[1]{\left\{\{#1\right\}_{S}}
\newcommand{\jumpS}[1]{\left[#1\right]_{S}}
\newcommand{\jump}[1]{\left[#1\right]}
\newcommand{\mean}[1]{\left\{#1\right\}}
\newcommand{\intS}{\int_{\Sides_{h}}}
\newcommand{\intSInt}{\int_{\SidesInt_{h}}}
\newcommand{\intSBound}{\int_{\SidesBound_{h}}}
%\newcommand{\intK}{\int_{\Cells_{h}}}
\newcommand{\REFLOC}{{\rm REF}}
\newcommand{\AFEM}[1]{\textbf{AFEM}($#1$)}

%
\newcommand{\tesfctu}{\delta\! u}
\newcommand{\udir}{u^{\rm D}}
\newcommand{\vdir}{v^{\rm D}}
\newcommand{\wdir}{w^{\rm D}}
\newcommand{\uh}{u_{h}}
\newcommand{\phih}{\phi_{h}}
\newcommand{\uin}{u^{\rm in}}
\newcommand{\uex}{u^{\rm ex}}
\newcommand{\uhin}{u^{\rm in}_{h}}
\newcommand{\uhex}{u^{\rm ex}_{h}}
\newcommand{\ubar}{\overline{u}}
\newcommand{\ustar}{u^{*}}
\newcommand{\sgn}[1]{\operatorname{sgn}(#1)}
%
\newcommand{\In}[1]{#1^{\rm in}}
\newcommand{\Ex}[1]{#1^{\rm ex}}
\newcommand{\InDe}[2]{#1^{{\rm in}_{#2}}}
\newcommand{\ExDe}[2]{#1^{{\rm ex}_{#2}}}

%---------------------------------------------------------


\newcommand{\mcM}{\mathcal M}

%-------------------------------------------
\begin{document}
%-------------------------------------------

\title{Convex functions}
\author{}
\maketitle
\tableofcontents

%
%==========================================
\section{Convex functions}\label{sec:}
%==========================================
%
%---------------------------------------
\begin{lemma}\label{lemma:}
%
For $f\in \mathcal F^{1,1}_{L}(X)$
%
\begin{equation}\label{eq:conv_lip}
%
\left\{
\begin{aligned}
\frac{t(1-t)}{2L}\norm{\nabla f(x_1)-\nabla f(x_2)}^2\le&(1-t) f(x_1) + t f(x_2) - f( (1-t)x_1 + t x_2) &\le& \frac{t(1-t)L}{2}\norm{x_1-x_2}^2\\
\frac{1}{2L} \norm{\nabla f(x)-\nabla f(x_0)}^2 \le& f(x) - f(x_0) - \scp{\nabla f(x_0)}{x-x_0} &\le& \frac{L}{2} \norm{x-x_0}^2\\
\frac1L \norm{\nabla f(x) -\nabla f(x_0)}^2\le& \scp{\nabla f(x) -\nabla f(x_0)}{x-x_0} &\le&  L \norm{x-x_0}^2
\end{aligned}
\right.
\end{equation}
%
\end{lemma}
%
%
%---------------------------------------
\begin{proof}
%
\begin{align*}
f(x) - f(x_0) - \scp{\nabla f(x_0)}{x-x_0} = \int_0^1 \scp{\nabla f(x_0+t(x-x_0))-\nabla f(x_0)}{x-x_0}
\le L \int_0^1t\,dt \norm{x-x_0}^2
\end{align*}
%
Let
%
\begin{align*}
g(y) = f(y) - \scp{\nabla f(x_0)}{y}
\end{align*}
%
Then $g\in \mathcal F^{1,1}_{L}(X)$ and $x_0$ is a minimizer of $g$, so with  $x=x_0+ \frac{1}{L}\nabla g(x)$
%
%
\begin{align*}
f(x_0) - f(x) -\scp{\nabla f(x_0)}{x_0-x} \le g(x_0) - g(x) \le  \scp{\nabla g(x)}{x_0-x}+\frac{L}{2} \norm{x-x_0}^2
= -\frac{L}{2} \norm{\nabla g(x)}^2
\end{align*}
%



\end{proof}
%
\begin{lemma}\label{lemma:}
%
For $f\in\mathcal S^{1,1}_{\mu,L}(X) := \SetDef{f\in \mathcal F^{1,1}_{L}(X)}{f(x) - f(x_0) - \scp{\nabla f(x_0)}{x-x_0}\ge \frac{\mu}{2} \norm{x-x_0}^2}$
%
%
\begin{equation}\label{eq:conv_lip_smooth}
%
\left\{
\begin{aligned}
\frac{1}{\mu+L} \norm{\nabla f(x) -\nabla f(x_0)}^2 + \frac{\mu L}{\mu+L}\norm{x-x_0}^2\le& \scp{\nabla f(x) -\nabla f(x_0)}{x-x_0}
\end{aligned}
\right.
\end{equation}
%
Let $g(x) := f(x) - \frac{\mu}{2}\norm{x}^2$. Then $\scp{\nabla g(x_1)-\nabla g(x_2)}{x_1-x_2} = \scp{\nabla f(x_1)-\nabla f(x_2)}{x_1-x_2} - \mu\scp{x_1-x_2}{x_1-x_2} \le (L-\mu)\norm{x_1-x_2}^2$, so $g\in \mathcal F^{1,1}_{L-\mu}(X)$. Then for $\mu<L$
%
\begin{align*}
&\scp{\nabla f(x_1)-\nabla f(x_2)}{x_1-x_2} - \mu\norm{x_1-x_2}^2 = \scp{\nabla g(x_1)-\nabla g(x_2)}{x_1-x_2}\ge \frac{1}{L-\mu}\norm{\nabla g(x_1)-\nabla g(x_2)}^2\\
\quad&= \frac{1}{L-\mu}\left(\norm{\nabla f(x_1)-\nabla f(x_2)}^2 -2\mu\scp{\nabla f(x_1)-\nabla f(x_2)}{x_1-x_2} + \mu^2\norm{x_1-x_2}^2\right)\\
\quad\Rightarrow\quad \\
&(L+\mu)\scp{\nabla f(x_1)-\nabla f(x_2)}{x_1-x_2} \ge \norm{\nabla f(x_1)-\nabla f(x_2)}^2 + (\mu^2+(L-\mu)\mu)\norm{x_1-x_2}^2.
\end{align*}
\end{lemma}
%
%
%==========================================
\section{FISTA smooth \cite{BeckTeboulle09}}\label{sec:}
%==========================================
%
Let
%
\begin{align*}
Q_{\alpha}(x,y) := f(y) + f'(y)(x-y) + \frac{\alpha}{2}\norm{x-y}^2
\end{align*}
%
We have by (\ref{eq:conv_lip})
%
\begin{equation}\label{eq:QLoverestimate}
f(x) \le Q_L(y,x)\quad \forall x,y\in X.
\end{equation}
%
and
%
\begin{equation}\label{eq:}
Q_{\alpha}(x,y) = f(y) - \frac{1}{2\alpha}\norm{\nabla f(y)}^2 + \frac{\alpha}{2}\norm{x-y+\frac{1}{\alpha} \nabla f(y)}^2
\end{equation}
%
Let
%
\begin{equation}\label{eq:}
p_{\alpha}(y) = \argmin\SetDef{Q_{\alpha}(x,y)}{x\in X}
\end{equation}
%
Then
%
\begin{equation}\label{eq:Qmin}
p_{\alpha}(y) = y - \frac{1}{\alpha} \nabla f(y),\quad Q_{\alpha}(p_{\alpha}(y),y) = f(y) - \frac{1}{2\alpha}\norm{\nabla f(y)}^2 = f(y)-\frac{\alpha}{2}\norm{p_{\alpha}(y)-y}^2.
\end{equation}
%
%
\begin{yellow}
\begin{algorithm}[H]
\caption{FISTA smooth} 
\label{algorithm:FISTAsmooth} 
%
Choose $x_0\in X$, $y_0=x_0$, $t_0=1$. Set $k=0$.
%
\begin{itemize}
\item[(1)] $x_{k+1} = p_L(y_k)$.
\item[(2)] $s_{k+1} = \frac{1+\sqrt{1+4s_{k}^2}}{2}$
\item[(3)] $y_{k+1} = x_{k+1} + \frac{s_{k}-1}{s_{k+1}}(x_{k+1}-x_{k})$
\item[(4)] Increment $k$ and go to (1).
\end{itemize}
%
\end{algorithm}
\end{yellow}
%
We have
%
\begin{equation}\label{eq:FITAtk}
s_{k}\ge 1,\quad s_{k+1}\ge s_{k}, \quad s_{k+1}^2 - s_{k+1} = s_{k}^2, \quad s_{k} \ge (k+1)/2.
\end{equation}

%
\begin{align*}
\left(\frac{s_{k}-1}{s_{k+1}}\right)^2 = \frac{s_{k+1}^2-s_{k+1}-2s_{k}+1}{s_{k+1}^2}\le1
\end{align*}
%
%
\begin{align*}
\beta_k = \frac{s_{k}-1}{s_{k+1}} = \frac{2s_{k}-2}{1+\sqrt{1+4s_{k}^2}}
= \frac{(s_{k}-1)(1-\sqrt{1+4s_{k}^2})}{-2s_{k}^2}
\end{align*}
%

%
%---------------------------------------
\begin{lemma}\label{lemma:FISTAdecrease}
%
\begin{equation}\label{eq:FISTAdecrease}
f(p_L(y))-f(x) \le   \scp{\nabla f(y)}{y-x} -\frac{1}{2L}\norm{\nabla f(y)}^2 \quad\forall x\in X.
\end{equation}
%
\end{lemma}
%
%---------------------------------------
\begin{proof}
%
Since by (\ref{eq:QLoverestimate}) $Q_L$ is an overestimate of $f$, we have 
%
\begin{align*}
f(p_L(y))\le Q_L(p_L(y), y)
\end{align*}
%
Then we have, together with (\ref{eq:Qmin}) and convexity
%
%
\begin{align*}
f(p_L(y))-f(x) \le& Q_L(p_L(y), y) -f(y) - \scp{\nabla f(y)}{x-y}
= \scp{\nabla f(y)}{y-x} -\frac{1}{2L}\norm{\nabla f(y)}^2 .
\end{align*}
%
\end{proof}
%
%
For $y=y_k$ and $x=x_k$, $x=x^*$ in (\ref{eq:FISTAdecrease}), and with $\Delta f_k := f(x_k)- f(x^*)$
%
\begin{align*}
\Delta f_{k+1} - \Delta f_{k}\le  \scp{\nabla f(y_k)}{y_k-x_k} - \frac{1}{2L}\norm{\nabla f(y_k)}^2 \\
\Delta f_{k+1} \le \scp{\nabla f(y_k)}{y_k-x^*} - \frac{1}{2L}\norm{\nabla f(y_k)}^2  
\end{align*}
%
so multiplying the first inequality with $s_{k}^2$ and the second with $s_{k+1}^2-s_{k}^2$
%
\begin{align*}
 s_{k+1}^2\Delta f_{k+1}-s_{k}^2 \Delta f_{k} \le 
  \scp{\nabla f(y_k)}{s_{k+1}^2y_k-s_{k}^2 x_k -(s_{k+1}^2-s_{k}^2) x^*}-\frac{s_{k+1}^2}{2L}\norm{\nabla f(y_k)}^2
\end{align*}
%
%
Now the condition
%
\begin{equation}\label{eq:}
s_{k+1}^2-s_{k}^2 =s_{k+1}
\end{equation}
%
implies
%
\begin{align*}
s_{k+1}^2 \Delta f_{k+1}-s_{k}^2 \Delta f_{k} \le& 
  \scp{s_{k+1}\nabla f(y_k)}{s_{k+1}y_k+(1-s_{k+1}) x_k - x^*}-\frac{L}{2}\norm{\frac{s_{k+1}}{L}\nabla f(y_k)}^2\\
=& \frac{L}{2}\left(2\scp{a_k}{b_k}-\norm{a_k}^2 \right) =
\frac{L}{2}\left(\norm{b_k}^2-\norm{b_k-a_k}^2 \right)\\
a_k:=& \frac{s_{k+1}}{L}\nabla f(y_k) = s_{k+1}(y_k-x_{k+1}),\quad
b_k :=  s_{k+1}y_k + (1-s_{k+1}) x_k -x^*
\end{align*}


Since
%
\begin{align*}
b_k-a_k =  s_{k+1}x_{k+1}+ (1-s_{k+1}) x_k - x^*
\end{align*}
%
%
\begin{align*}
b_k-a_k = b_{k+1} \quad\Leftrightarrow\quad s_{k+1}x_{k+1}+ (1-s_{k+1}) x_k  =  t_{k+2}y_{k+1}+(1-t_{k+2}) x_{k+1},
\end{align*}
%
or
%
\begin{align*}
y_{k+1} = x_{k+1} + \frac{(s_{k+1}-1)}{t_{k+2}}(x_{k+1}- x_k).
\end{align*}
%
Now from 
%
\begin{equation}\label{eq:}
\frac{2s_{k+1}^2}{L} \Delta f_{k+1} - \frac{2s_{k}^2}{L} \Delta f_k \le  \norm{b_{k}}^2 - \norm{b_{k+1}}^2
\end{equation}
%
it follows that for any $k\ge1$
%
\begin{align*}
\frac{2s_{k+1}^2}{L} \Delta f_{k+1} \le \frac{2s_{k+1}^2}{L} \Delta f_{k+1} + \norm{b_{k+1}}^2 \le \frac{2s_{k}^2}{L} \Delta f_{k} + \norm{b_{k}}^2
\le \frac{2t_{0}^2}{L} \Delta f_{0} + \norm{b_{0}}^2
\end{align*}
%
and with (\ref{eq:FITAtk})
%
\begin{equation}\label{eq:}
f(x_k) - f^* \le \frac{1}{s_{k}^2}\left( f(x_0) - f^*  + \frac{L}{2}\norm{b_{0}}^2\right).
\end{equation}
%
%
%
%==========================================
\section{FISTA with backtracking ???}\label{sec:}
%==========================================
%
%
%
%
%==========================================
\section{Nesterov acceleration}\label{sec:}
%==========================================
%
%
\begin{yellow}
\begin{algorithm}[H]
\caption{AGD fixed} 
\label{algorithm:AGDfixed} 
%
Choose $x_0\in X$, $\alpha=\sqrt{\kappa_f}$, $\beta=\frac{\alpha-1}{\alpha+1}$. Set $k=0$.
%
\begin{itemize}
\item[(1)] $x_{k+1} = y_k- \frac1L \nabla f(y_k)$.
\item[(3)] $y_{k+1} = x_{k+1} + \beta(x_{k+1}-x_{k})$
\item[(4)] Increment $k$ and go to (1).
\end{itemize}
%
\end{algorithm}
\end{yellow}
%



Let us start with, for any $x\in X$,
%
%
\begin{align*}
%
\left\{
\begin{aligned}
f(x_{k+1}) \le&   f(y_k)  -\frac{1}{2L}\norm{\nabla f(y_k)}^2\\
f(x) \ge& f(y_k) + \scp{\nabla f(y_k)}{x-y_k} + \frac{\mu}{2}\norm{x-y_k}^2
\end{aligned}
\right.\\
\quad\Rightarrow\quad 
f(x_{k+1})-f(x)\le \scp{\nabla f(y_k)}{y_k-x} -\frac{1}{2L}\norm{\nabla f(y_k)}^2  -\frac{\mu}{2}\norm{x-y_k}^2
%
\end{align*}
%
Setting $\Delta f_k:= f(x_k)-f^*$ we then have with $0<\theta<1$ and using $2ab -a^2 = b^2-(a-b)^2$
%
\begin{align*}
\Delta f_{k+1} -(1-\theta)\Delta f_k \le& \scp{\nabla f(y_k)}{y_k-(1-\theta)x_k -\theta x^*} -\frac{1}{2L}\norm{\nabla f(y_k)}^2  -\frac{\theta\mu}{2}\norm{x^*-y_k}^2\\
=& \frac{L}{2}\left( \norm{y_k-(1-\theta)x_k -\theta x^*}^2 - \norm{x_{k+1}-(1-\theta)x_k -\theta x^*}^2\right) -\frac{\theta\mu}{2}\norm{x^*-y_k}^2
\end{align*}
%
With $\theta=\frac{1-\beta}{1+\beta}$, such that $\beta=\frac{1-\theta}{1+\theta}$ and 
$\frac{1-\theta}{\theta\beta}-1  = \frac{1+\theta}{\theta}-1 = \frac{1}{\theta}$
%
\begin{align*}
z_k :=& \frac{1}{\theta}x_k - \frac{1-\theta}{\theta} x_{k-1} = x_k  + \frac{1-\theta}{\theta} (x_k-x_{k-1})
= y_k + \left(  \frac{1-\theta}{\theta}-\beta\right) (x_k-x_{k-1})\\ 
=& y_k + \left(\frac{1-\theta}{\beta\theta}-1\right)(y_k-x_k)
= y_k + \frac{1}{\theta}(y_k-x_k)
\end{align*}
%
and
%
\begin{align*}
\frac{1}{\theta}y_k - \frac{1-\theta}{\theta} x_{k} = x_k +\frac{1}{\theta}(y_k-x_k)=  z_k + x_k-y_k
\end{align*}
%
Putting these together, we find
%
\begin{align*}
\Delta f_{k+1} -(1-\theta)\Delta f_k \le& \frac{L}{2}\theta^2\left( \norm{z_k- x^* + x_k-y_k}^2 - \norm{z_{k+1} - x^*}^2\right) -\frac{\theta\mu}{2}\norm{x^*-y_k}^2
\end{align*}
%
Now we have 
%
\begin{align*}
2\scp{z_k- x^*}{ x_k-y_k} =& 2\theta\scp{z_k- x^*}{ y_k-z_k} = \theta\left( \norm{y_k- x^*}^2 - \norm{z_k-x^*}^2 - \norm{y_k- z_k}^2 \right)\\
=&\theta\left( \norm{y_k- x^*}^2 - \norm{z_k-x^*}^2 - \frac{1}{\theta^2}\norm{y_k- x_k}^2 \right)
\end{align*}
%
such that
%
\begin{align*}
 \norm{z_k- x^* + x_k-y_k}^2 =&  \norm{z_k- x^* }^2 + 2\scp{z_k- x^*}{x_k-y_k} + \norm{x_k-y_k}^2\\
 =&  \norm{z_k- x^* }^2 - \theta\left(\norm{z_k- x^*}^2-\norm{y_k- x^*}^2\right) - \frac{1}{\theta}\norm{y_k-x_k}^2 + \norm{x_k-y_k}^2\\
 =&  (1-\theta)\norm{z_k- x^* }^2 + \theta\norm{y_k- x^*}^2 - \frac{1-\theta}{\theta}\norm{y_k-x_k}^2 
\end{align*}
%
and we get
%
\begin{align*}
\theta\left(\Delta f_{k+1} + \frac{L}{2}\theta^2\norm{z_{k+1} - x^*}^2\right) \le& (1-\theta)\left(\Delta f_k-\Delta f_{k+1}\right)  + \frac{L}{2}\theta^2(1-\theta)\left( \norm{z_k- x^* }^2 - \norm{z_{k+1} - x^*}^2\right)\\
&+ \frac{L}{2}\theta^3\norm{y_k- x^*}^2 - 2L\theta^2\frac{1-\theta}{\theta}\norm{y_k-x_k}^2  -\frac{\theta\mu}{2}\norm{x^*-y_k}^2
\end{align*}
%
For 
%
\begin{align*}
\frac{L}{2}\theta^2 \le \frac{\mu}{2} \quad\Leftrightarrow\quad \theta\le \frac{1}{\sqrt{\kappa}}
\end{align*}
%
we have with $a_k:=\Delta f_{k} + 2L\theta^2\norm{z_{k} - x^*}^2$
%
\begin{align*}
\sum_{k=n+1}^{\infty} a_k \le \frac{1-\theta}{\theta} a_n,
\end{align*}
%
so with $S_n:= \sum_{k=n}^{\infty} a_k$
%
\begin{align*}
S_{n+1} \le \frac{1-\theta}{\theta} \left(S_n-S_{n+1}\right) \quad\Rightarrow\quad S_{n+1} \le (1-\theta )S_n
\quad\Rightarrow\quad  S_{n+k} \le (1-\theta )^k S_n
\end{align*}
%
or
%
\begin{equation}\label{eq:}
a_{n+k} \le S_{n+k} \le (1-\theta )^{k-1} S_{n+1}\le \frac{(1-\theta )^{k}}{\theta} a_{n}
\end{equation}
%









%-----------------------------------------------
\printbibliography
%-----------------------------------------------
%
%-------------------------------------------
\end{document}      
%-------------------------------------------Wi\